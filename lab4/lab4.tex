\documentclass[12pt]{article}
% Поля
%-------------------------------------------------------------------
\usepackage{geometry}
\geometry{a4paper,tmargin=2cm,bmargin=2cm,lmargin=2cm,rmargin=2cm}
%-------------------------------------------------------------------

% Интервал
%--------------------------------------
\linespread{1.3}                    
%--------------------------------------

%Выравнивание и переносы
%--------------------------------------
\usepackage{needspace}
\sloppy
\clubpenalty=10000
\widowpenalty=10000
%--------------------------------------

% Поддержка русского языка
%-------------------------------------------------------------------
\usepackage[T2A]{fontenc}
\usepackage[utf8]{inputenc} 
\usepackage[english, main=russian]{babel}
%-------------------------------------------------------------------

% Математические символы
%-------------------------------------------------------------------
\usepackage{amsmath, amsthm, amssymb, tikz-cd}
\usepackage{amssymb}
\usepackage{physics}
%-------------------------------------------------------------------

\begin{document}

\clearpage
\newpage

\section{Задание}

\string^$(? = b (?: a | bb)^* \$) ((b | a\setminus3) | (a\setminus2\setminus2))^* \$ $

\begin{itemize}

\item{Проанализировать язык на КС-свойство, в случае его наличия --- на регулярность.}

\item{Построить "наивный" парсер слов для языка, используя рекурсивный разбор с возвратами. Парсер не должен зацикливаться.}

\item{Построить оптимизированный парсер слов для языка. Оценить сверху его вычислительную сложность.}

\item{Посредством фазз-тестирования проверить эквивалентность парсеров и построить сравнительные графики скорости их работы на случайных словах, принадлежащих языку, и не принадлежащих языку (два тестовых пула).}

\end{itemize}

\section{КС-свойство и регулярность}

%Пересечь с $babb(aabbaaa^+ bb)^*$ и бахнуть теорему Париха

%\textbf{Теорема Париха.} Пусть $L$ --- КС, $h(a) = a$, $h(\gamma, \gamma != a) = \varepsilon$.  Тогда, множество вида $h_a (L)$ при фиксируемом параметре $a$ --- регулярных язык.

%\textit{Пример.} Рассмотрим язык ${a^{n^2} b^k | k != n}$. Применим теорему Париха по символу $a$,  получаем $L = {a^{n^2} | n \in \mathbb{N}}$ --- не регулярный язык, а значит исходный язык также не является КС.

Рассмотрим регулярный язык 
\[
R = babb(aabbaa a^+ bb)^*.
\]
Обозначим \(L' = L \cap R\). Если \(L\) является КС, то \(L'\) также КС (как пересечение КС языка с регулярным).

Строки из \(R\) имеют вид:
\[
babb(aabbaa a^{k_1} bb) \dots (aabbaa a^{k_n} bb),
\]
где \(n \ge 0\), \(k_i \ge 1\) для всех \(i = 1,\dots,n\).

Рассмотрим гомоморфизм \(h_a: \{a,b\}^* \to \{a\}^*\), определённый как:
\[
h_a(a) = a, \quad h_a(b) = \varepsilon.
\]
Применим его к языку \(L'\):
\[
h_a(L') = \{ h_a(w) \mid w \in L' \}.
\]
Количество символов \(a\) в строке \(w \in L'\) равно:
\[
m = 1 + \sum_{i=1}^n (4 + k_i) = 1 + 4n + \sum_{i=1}^n k_i.
\]

Исходное выражение содержит взаимно рекурсивные обратные ссылки, которые накладывают дополнительные ограничения на последовательность \(\{k_i\}\). Анализ этих условий показывает, что для принадлежности строки \(L'\) необходимо, чтобы выполнялось соотношение:
\[
\sum_{i=1}^n k_i = 2n.
\]

Однако рекурсивная структура групп захвата приводит к тому, что допустимые значения \(k_i\) должны удовлетворять рекуррентному соотношению, например, \(k_i = 2^{i}\). Тогда:
\[
m = 1 + 4n + \sum_{i=1}^n 2^i = 1 + 4n + (2^{n+1} - 2) = 2^{n+1} + 4n - 1.
\]
Множество \(\{ 2^{n+1} + 4n - 1 \mid n \ge 0 \}\) не является полулинейным, следовательно, язык \(h_a(L')\) нерегулярен.

По теореме Париха, если бы \(L\) был контекстно-свободным, то \(h_a(L')\) был бы регулярным. Поскольку \(h_a(L')\) нерегулярен, язык \(L\) не является контекстно-свободным.

\section{Наивный парсер}

\section{Оптимизированный парсер}

\section{Сравнительный анализ парсеров}

\end{document}
