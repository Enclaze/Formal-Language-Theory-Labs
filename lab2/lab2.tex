\documentclass[12pt]{article}
% Поля
%-------------------------------------------------------------------
\usepackage{geometry}
\geometry{a4paper,tmargin=2cm,bmargin=2cm,lmargin=2cm,rmargin=2cm}
%-------------------------------------------------------------------

% Интервал
%--------------------------------------
\linespread{1}                    
%--------------------------------------

%Выравнивание и переносы
%--------------------------------------
\usepackage{needspace}
\sloppy
\clubpenalty=10000
\widowpenalty=10000
%--------------------------------------

% Поддержка русского языка
%-------------------------------------------------------------------
\usepackage[T2A]{fontenc}
\usepackage[utf8]{inputenc} 
\usepackage[english, main=russian]{babel}
%-------------------------------------------------------------------

% Математические символы
%-------------------------------------------------------------------
\usepackage{amsmath, amsthm, amssymb, tikz-cd}
\usepackage{amssymb}
\usepackage{physics}
%-------------------------------------------------------------------
\usepackage{float}
\usepackage{adjustbox}
\usepackage{booktabs}
\usepackage{rotating}
\usepackage{array}
\newcolumntype{C}{>{\centering\arraybackslash}p{0.8em}}


\begin{document}

\clearpage
\newpage

\section{Задание}
Дано следующее академическое регулярное выражение:

\begin{center}
$((aa|ab|cc)^*aba(aaa|bcc)^*)^*((abac|(cc)^*)(b|ca))^*$
\end{center}

\noindent По имеющемуся академическому регулярному выражению построить
\begin{itemize}
    
    \item{минимальный ДКА, распознающий его язык (минимальность обосновать таблицей классов эквивалентности)}
    
    \item{возможно малый НКА, распознающий его язык. Возможно малый переключающийся (с конъюнкцией) КА, распознающий его язык. Частично обосновать таблицами множеств классов эквивалентности.}
    
    \item{расширенное регулярное выражение, распознающее тот же язык. В расширенном выражении можно использовать:
    \begin{itemize}

        \item{wildcard-операцию для замены произвольного алфавита;}
         
        \item{положительную итерацию $\tau^+$ и опцию $\tau?.\tau^+ = \tau\tau^*,\tau? = (\tau|\varepsilon)$;}
        
        \item{операции предпросмотра $\tau_0(?=\tau_1)\tau_2 \equiv \tau_0((\tau_1.^*)\cap\tau_2)$ и ретроспективной проверки $\tau_0(?<= \tau_1)\tau_2 \equiv (\tau_0 \cup (\tau_1.^*))\tau_2$, а также их отрицательные версии $\tau_0(?!\tau_1)\tau_2 \equiv \tau_0(\overline{(\tau_1.^*)}\cap \tau_2)$ и $\tau_0(?<!\tau_1)\tau_2 \equiv (\tau_0 \cap \overline{(\tau_1.^*)})\tau_2$;}
        
        \item{классы букв $[c_1\dots c_k] \equiv (c_1|c_2|\dots|c_k)$ и их дополнения $[\string^ c_1\dots c_k]$;}
        
        \item{(обязательно) маркеры начала и конца выражения \string^ и \$.}
    \end{itemize}
    }
\end{itemize}

Провести автоматическое тестирование предполагаемой эквивалентности построенных распознавателей. Тем самым необходимо построить алгоритмы, определяющие принадлежность слова языку академического регулярного выражения, ДКА, НКА и ПКА.

Эквивалентность расширенного регулярного выражения тестировать не обязательно (на дополнительный балл). Если не делать, то для расширенного регулярного выражения достаточно описать содержательно (полуформально), почему оно должно распознавать тот же самый язык.

Требуется только фазз-тестирование эквивалентности: строится случайное слово $\omega$ и проверяется, принадлежит ли оно языкам регулярного выражения, ДКА, НКА и ПКА согласованно.


\section{Минимальный ДКА}

\begin{figure}[H]
    \centering
    \begin{adjustbox}{width=\textwidth}
    \def \globalscale {1.000000}
\begin{tikzpicture}[y=1cm, x=1cm, yscale=\globalscale,xscale=\globalscale, every node/.append style={scale=\globalscale}, inner sep=0pt, outer sep=0pt]
  \begin{scope}[shift={(0.1411, -38.3046)}]
    \path[fill=white] (-0.1411, 38.3117) -- (-0.1411, 76.7574) -- (88.5567, 76.7574) -- (88.5567, 38.3117) -- (-0.1411, 38.3117) -- (-0.1411, 38.3117)-- cycle;;



    \path[draw=black] (4.119, 60.3956) ellipse (0.7676cm and 0.7676cm);



    \path[draw=black] (4.119, 60.3956) ellipse (0.9088cm and 0.9088cm);



    \node[anchor=south] (text3402) at (4.119, 60.2474){L7};



    \path[draw=black] (1.9001, 60.3956).. controls (2.1819, 60.3956) and (2.498, 60.3956) .. (2.8003, 60.3956);



    \path[draw=black,fill=black] (2.7933, 60.519) -- (3.1461, 60.3956) -- (2.7933, 60.2721) -- (2.7933, 60.519) -- (2.7933, 60.519)-- cycle;;



    \path[draw=black] (70.8029, 58.2789) ellipse (0.7676cm and 0.7676cm);



    \path[draw=black] (70.8029, 58.2789) ellipse (0.9088cm and 0.9088cm);



    \node[anchor=south] (text2035) at (70.8029, 58.1307){L1};



    \path[draw=black] (70.4956, 59.1517).. controls (70.4744, 59.515) and (70.5767, 59.8226) .. (70.8029, 59.8226).. controls (70.9373, 59.8226) and (71.0279, 59.7143) .. (71.0748, 59.5499);



    \path[draw=black,fill=black] (71.1973, 59.5672) -- (71.1055, 59.2049) -- (70.9514, 59.5454) -- (71.1973, 59.5672) -- (71.1973, 59.5672)-- cycle;;



    \node[anchor=south] (text8983) at (70.8029, 59.9708){b};



    \path[draw=black] (74.1391, 63.9233) ellipse (0.9003cm and 0.9003cm);



    \node[anchor=south] (text4448) at (74.1391, 63.7752){q33};



    \path[draw=black] (71.1299, 59.1294).. controls (71.3969, 59.8572) and (71.8287, 60.9191) .. (72.347, 61.7714).. controls (72.6027, 62.1923) and (72.9336, 62.6195) .. (73.2395, 62.9814);



    \path[draw=black,fill=black] (73.1397, 63.0548) -- (73.4642, 63.2404) -- (73.3263, 62.8929) -- (73.1397, 63.0548) -- (73.1397, 63.0548)-- cycle;;



    \node[anchor=south] (text8426) at (72.4563, 62.2363){a};



    \path[draw=black] (74.1391, 59.055) ellipse (0.938cm and 0.938cm);



    \node[anchor=south] (text9378) at (74.1391, 58.9068){L25};



    \path[draw=black] (71.7264, 58.2295).. controls (71.9988, 58.2334) and (72.2976, 58.257) .. (72.566, 58.3212).. controls (72.6927, 58.3516) and (72.8211, 58.3932) .. (72.947, 58.4412);



    \path[draw=black,fill=black] (72.8916, 58.5516) -- (73.2642, 58.5787) -- (72.9897, 58.3251) -- (72.8916, 58.5516) -- (72.8916, 58.5516)-- cycle;;



    \node[anchor=south] (text4951) at (72.4563, 58.4694){c};



    \path[draw=black] (10.3099, 65.8636) ellipse (0.7676cm and 0.7676cm);



    \node[anchor=south] (text8767) at (10.3099, 65.7154){L2};



    \path[draw=black] (13.505, 64.4525) ellipse (0.938cm and 0.938cm);



    \node[anchor=south] (text2945) at (13.505, 64.3043){L11};



    \path[draw=black] (10.7791, 65.2498).. controls (11.0183, 64.9644) and (11.3422, 64.6536) .. (11.7126, 64.4948).. controls (11.8512, 64.4356) and (12.0019, 64.3975) .. (12.1543, 64.3745);



    \path[draw=black,fill=black] (12.1606, 64.498) -- (12.5039, 64.3498) -- (12.143, 64.2518) -- (12.1606, 64.498) -- (12.1606, 64.498)-- cycle;;



    \node[anchor=south] (text1983) at (11.8223, 64.643){a};



    \path[draw=black] (47.9143, 68.5094) ellipse (0.9003cm and 0.9003cm);



    \node[anchor=south] (text9332) at (47.9143, 68.3613){q11};



    \path[draw=black] (10.3269, 66.6425).. controls (10.3272, 68.5444) and (10.6151, 73.2719) .. (13.4698, 73.2719).. controls (13.4698, 73.2719) and (13.4698, 73.2719) .. (30.1925, 73.2719).. controls (33.988, 73.2719) and (34.9366, 73.3425) .. (38.7322, 73.3425).. controls (38.7322, 73.3425) and (38.7322, 73.3425) .. (44.3022, 73.3425).. controls (46.0816, 73.3425) and (47.0775, 71.1994) .. (47.5439, 69.7674);



    \path[draw=black,fill=black] (47.6603, 69.8091) -- (47.6462, 69.4358) -- (47.4243, 69.7364) -- (47.6603, 69.8091) -- (47.6603, 69.8091)-- cycle;;



    \node[anchor=south] (text1486) at (28.3333, 73.4201){c};



    \path[draw=black] (55.2651, 53.2342) ellipse (0.7676cm and 0.7676cm);



    \node[anchor=south] (text7666) at (55.2651, 53.086){L3};



    \path[draw=black] (56.0331, 53.0871).. controls (57.773, 52.7667) and (62.2755, 52.1035) .. (65.8657, 53.0225).. controls (67.6039, 53.4674) and (68.1845, 53.7725) .. (69.2591, 55.2097).. controls (69.6789, 55.7713) and (70.0324, 56.4681) .. (70.2917, 57.0558);



    \path[draw=black,fill=black] (70.1756, 57.0981) -- (70.4271, 57.3747) -- (70.4028, 57.0015) -- (70.1756, 57.0981) -- (70.1756, 57.0981)-- cycle;;



    \node[anchor=south] (text8422) at (62.1344, 52.7456){a};



    \path[draw=black] (58.7703, 50.9411) ellipse (0.7676cm and 0.7676cm);



    \path[draw=black] (58.7703, 50.9411) ellipse (0.9088cm and 0.9088cm);



    \node[anchor=south] (text2329) at (58.7703, 50.7929){L4};



    \path[draw=black] (55.8281, 52.7004).. controls (56.1439, 52.3998) and (56.5637, 52.0312) .. (56.9796, 51.7596).. controls (57.1546, 51.6453) and (57.3479, 51.5384) .. (57.5409, 51.4417);



    \path[draw=black,fill=black] (57.5906, 51.5546) -- (57.857, 51.2925) -- (57.4851, 51.3313) -- (57.5906, 51.5546) -- (57.5906, 51.5546)-- cycle;;



    \node[anchor=south] (text7073) at (57.1031, 51.9077){c};



    \path[draw=black] (62.1344, 44.6264) ellipse (0.7676cm and 0.7676cm);



    \node[anchor=south] (text1076) at (62.1344, 44.4782){L9};



    \path[draw=black] (59.2226, 50.1414).. controls (59.8181, 48.9994) and (60.912, 46.9022) .. (61.5657, 45.6487);



    \path[draw=black,fill=black] (61.6673, 45.7214) -- (61.7209, 45.3513) -- (61.4482, 45.6071) -- (61.6673, 45.7214) -- (61.6673, 45.7214)-- cycle;;



    \node[anchor=south] (text6028) at (60.4375, 48.1214){c};



    \path[draw=black] (62.1344, 49.1419) ellipse (0.938cm and 0.938cm);



    \node[anchor=south] (text5216) at (62.1344, 48.9938){L15};



    \path[draw=black] (59.5859, 50.5146).. controls (59.9913, 50.2934) and (60.4943, 50.0186) .. (60.9438, 49.7731);



    \path[draw=black,fill=black] (60.9925, 49.887) -- (61.2429, 49.6097) -- (60.8739, 49.6704) -- (60.9925, 49.887) -- (60.9925, 49.887)-- cycle;;



    \node[anchor=south] (text8229) at (60.4375, 50.2437){a};



    \path[draw=black] (64.7866, 54.4336) ellipse (0.938cm and 0.938cm);



    \path[draw=black] (64.7866, 54.4336) ellipse (1.0791cm and 1.0791cm);



    \node[anchor=south] (text7734) at (64.7866, 54.2854){L16};



    \path[draw=black] (59.5708, 51.3905).. controls (60.56, 51.9716) and (62.296, 52.9915) .. (63.4838, 53.6889);



    \path[draw=black,fill=black] (63.4203, 53.7951) -- (63.7872, 53.8674) -- (63.5455, 53.582) -- (63.4203, 53.7951) -- (63.4203, 53.7951)-- cycle;;



    \node[anchor=south] (text7383) at (60.4375, 52.1074){b};



    \path[draw=black] (16.7277, 58.1731) ellipse (0.7676cm and 0.7676cm);



    \node[anchor=south] (text4586) at (16.7277, 58.0249){L5};



    \path[draw=black] (19.8935, 52.2817) ellipse (0.7676cm and 0.7676cm);



    \path[draw=black] (19.8935, 52.2817) ellipse (0.9088cm and 0.9088cm);



    \node[anchor=south] (text3058) at (19.8935, 52.1335){L6};



    \path[draw=black] (17.1182, 57.4971).. controls (17.6392, 56.5058) and (18.6175, 54.6435) .. (19.2532, 53.4335);



    \path[draw=black,fill=black] (19.359, 53.4977) -- (19.4137, 53.128) -- (19.1403, 53.3827) -- (19.359, 53.4977) -- (19.359, 53.4977)-- cycle;;



    \node[anchor=south] (text6144) at (18.24, 55.5523){a};



    \path[draw=black] (16.8032, 58.9478).. controls (16.958, 61.5343) and (17.6241, 69.7089) .. (19.8582, 69.7089).. controls (19.8582, 69.7089) and (19.8582, 69.7089) .. (35.4235, 69.7089).. controls (39.8921, 69.7089) and (41.2651, 73.1524) .. (45.3461, 71.3317).. controls (46.1324, 70.981) and (46.7762, 70.2518) .. (47.2182, 69.6228);



    \path[draw=black,fill=black] (47.3198, 69.693) -- (47.413, 69.3311) -- (47.1145, 69.5558) -- (47.3198, 69.693) -- (47.3198, 69.693)-- cycle;;



    \node[anchor=south] (text7975) at (31.9952, 69.8571){c};



    \path[draw=black] (19.9044, 51.3556).. controls (19.902, 49.2619) and (20.2159, 44.3442) .. (23.2223, 44.3442).. controls (23.2223, 44.3442) and (23.2223, 44.3442) .. (28.3686, 44.3442).. controls (31.9204, 44.3442) and (32.6679, 43.2255) .. (36.1467, 42.5097).. controls (37.6343, 42.2035) and (38.016, 42.1788) .. (39.5122, 41.9171).. controls (40.7928, 41.693) and (41.0937, 41.4161) .. (42.3937, 41.4161).. controls (42.3937, 41.4161) and (42.3937, 41.4161) .. (58.8056, 41.4161).. controls (60.0675, 41.4161) and (61.0263, 42.6392) .. (61.5756, 43.5758);



    \path[draw=black,fill=black] (61.4669, 43.6347) -- (61.7463, 43.8831) -- (61.6828, 43.5148) -- (61.4669, 43.6347) -- (61.4669, 43.6347)-- cycle;;



    \node[anchor=south] (text8664) at (40.5913, 42.0652){c};



    \path[draw=black] (23.2576, 52.2817) ellipse (0.938cm and 0.938cm);



    \node[anchor=south] (text6151) at (23.2576, 52.1335){L12};



    \path[draw=black] (20.8114, 52.2817).. controls (21.1466, 52.2817) and (21.5357, 52.2817) .. (21.9026, 52.2817);



    \path[draw=black,fill=black] (21.8966, 52.4051) -- (22.2493, 52.2817) -- (21.8966, 52.1582) -- (21.8966, 52.4051) -- (21.8966, 52.4051)-- cycle;;



    \node[anchor=south] (text5390) at (21.5611, 52.4298){a};



    \path[draw=black] (44.2669, 55.2097) ellipse (0.938cm and 0.938cm);



    \path[draw=black] (44.2669, 55.2097) ellipse (1.0791cm and 1.0791cm);



    \node[anchor=south] (text4895) at (44.2669, 55.0616){L13};



    \path[draw=black] (20.1831, 51.4015).. controls (20.5987, 50.1706) and (21.5501, 48.0836) .. (23.2223, 48.0836).. controls (23.2223, 48.0836) and (23.2223, 48.0836) .. (28.3686, 48.0836).. controls (34.3641, 48.0836) and (40.6033, 52.3716) .. (43.1003, 54.295);



    \path[draw=black,fill=black] (43.0174, 54.387) -- (43.3716, 54.5066) -- (43.1691, 54.1923) -- (43.0174, 54.387) -- (43.0174, 54.387)-- cycle;;



    \node[anchor=south] (text4523) at (31.9952, 48.7087){b};



    \path[draw=black] (4.1395, 61.3149).. controls (4.1099, 64.6649) and (4.2316, 76.0236) .. (7.2499, 76.0236).. controls (7.2499, 76.0236) and (7.2499, 76.0236) .. (42.4642, 76.0236).. controls (47.6786, 76.0236) and (48.1782, 71.755) .. (53.3922, 71.755).. controls (53.3922, 71.755) and (53.3922, 71.755) .. (60.4728, 71.755).. controls (65.7458, 71.755) and (67.1424, 68.6118) .. (69.2591, 63.7822).. controls (69.8648, 62.4004) and (70.2836, 60.7148) .. (70.5238, 59.5761);



    \path[draw=black,fill=black] (70.6438, 59.6057) -- (70.5933, 59.2353) -- (70.4017, 59.5563) -- (70.6438, 59.6057) -- (70.6438, 59.6057)-- cycle;;



    \node[anchor=south] (text3938) at (35.3882, 76.1718){b};



    \path[draw=black] (7.2852, 60.3956) ellipse (0.7676cm and 0.7676cm);



    \node[anchor=south] (text6780) at (7.2852, 60.2474){L8};



    \path[draw=black] (5.0331, 60.3956).. controls (5.3672, 60.3956) and (5.7527, 60.3956) .. (6.1076, 60.3956);



    \path[draw=black,fill=black] (6.0988, 60.519) -- (6.4516, 60.3956) -- (6.0988, 60.2721) -- (6.0988, 60.519) -- (6.0988, 60.519)-- cycle;;



    \node[anchor=south] (text8958) at (5.7725, 60.5437){a};



    \path[draw=black] (4.1854, 59.4808).. controls (4.3381, 55.3279) and (5.0906, 38.4528) .. (7.2499, 38.4528).. controls (7.2499, 38.4528) and (7.2499, 38.4528) .. (49.7875, 38.4528).. controls (54.3186, 38.4528) and (55.7795, 38.6151) .. (59.6791, 40.9222).. controls (60.6654, 41.5057) and (61.3237, 42.6667) .. (61.7015, 43.5342);



    \path[draw=black,fill=black] (61.5869, 43.58) -- (61.8352, 43.8594) -- (61.8155, 43.4862) -- (61.5869, 43.58) -- (61.5869, 43.58)-- cycle;;



    \node[anchor=south] (text3871) at (31.9952, 38.6009){c};



    \path[draw=black] (7.6824, 61.0662).. controls (8.1947, 62.0141) and (9.1345, 63.7533) .. (9.7307, 64.8568);



    \path[draw=black,fill=black] (9.6167, 64.9062) -- (9.8933, 65.1577) -- (9.834, 64.7887) -- (9.6167, 64.9062) -- (9.6167, 64.9062)-- cycle;;



    \node[anchor=south] (text2322) at (8.7976, 63.4418){a};



    \path[draw=black] (26.651, 53.9397) ellipse (0.938cm and 0.938cm);



    \node[anchor=south] (text8211) at (26.651, 53.7916){L18};



    \path[draw=black] (7.3872, 59.6173).. controls (7.5629, 58.1148) and (8.1809, 54.9628) .. (10.2747, 54.9628).. controls (10.2747, 54.9628) and (10.2747, 54.9628) .. (23.2929, 54.9628).. controls (24.0228, 54.9628) and (24.8042, 54.7345) .. (25.4258, 54.4936);



    \path[draw=black,fill=black] (25.4617, 54.6125) -- (25.7411, 54.3645) -- (25.3679, 54.3842) -- (25.4617, 54.6125) -- (25.4617, 54.6125)-- cycle;;



    \node[anchor=south] (text8420) at (16.7277, 55.1109){b};



    \path[draw=black] (62.8124, 45.0067).. controls (64.2267, 45.8787) and (67.6123, 48.1725) .. (69.2591, 51.0469).. controls (70.3284, 52.9135) and (70.6441, 55.4387) .. (70.7348, 56.9581);



    \path[draw=black,fill=black] (70.6109, 56.9591) -- (70.7521, 57.3052) -- (70.8579, 56.9468) -- (70.6109, 56.9591) -- (70.6109, 56.9591)-- cycle;;



    \node[anchor=south] (text4499) at (67.4391, 49.8888){a};



    \path[draw=black] (67.4391, 44.6264) ellipse (0.938cm and 0.938cm);



    \node[anchor=south] (text9985) at (67.4391, 44.4782){L10};



    \path[draw=black] (62.8237, 44.2743).. controls (63.0901, 44.1498) and (63.4055, 44.0256) .. (63.7074, 43.9632).. controls (64.6469, 43.7691) and (64.9263, 43.7691) .. (65.8657, 43.9632).. controls (65.9846, 43.9875) and (66.1053, 44.0217) .. (66.2242, 44.0616);



    \path[draw=black,fill=black] (66.1705, 44.1731) -- (66.5441, 44.1851) -- (66.2594, 43.9427) -- (66.1705, 44.1731) -- (66.1705, 44.1731)-- cycle;;



    \node[anchor=south] (text6882) at (64.7866, 44.1113){c};



    \path[draw=black] (68.0706, 45.3302).. controls (68.4738, 45.8459) and (68.9818, 46.5808) .. (69.2591, 47.3146).. controls (70.5111, 50.6271) and (70.7422, 54.8527) .. (70.7736, 56.9602);



    \path[draw=black,fill=black] (70.6501, 56.9595) -- (70.7771, 57.3109) -- (70.897, 56.957) -- (70.6501, 56.9595) -- (70.6501, 56.9595)-- cycle;;



    \node[anchor=south] (text4234) at (69.1356, 47.4627){b};



    \path[draw=black] (66.4877, 44.6264).. controls (65.605, 44.6264) and (64.2669, 44.6264) .. (63.3144, 44.6264);



    \path[draw=black,fill=black] (63.3257, 44.5029) -- (62.973, 44.6264) -- (63.3257, 44.7499) -- (63.3257, 44.5029) -- (63.3257, 44.5029)-- cycle;;



    \node[anchor=south] (text1844) at (64.7866, 44.7746){c};



    \path[draw=black] (67.0934, 43.7444).. controls (66.4528, 42.1009) and (64.8169, 38.735) .. (62.1697, 38.735).. controls (57.0678, 38.735) and (57.0678, 38.735) .. (57.0678, 38.735).. controls (53.7979, 38.735) and (53.0571, 39.7933) .. (49.7875, 39.7933).. controls (38.7322, 39.7933) and (38.7322, 39.7933) .. (38.7322, 39.7933).. controls (29.8503, 39.7933) and (26.1676, 37.6375) .. (18.9847, 42.8625).. controls (15.6379, 45.297) and (14.0808, 58.7629) .. (13.661, 63.1035);



    \path[draw=black,fill=black] (13.5382, 63.0901) -- (13.6278, 63.4531) -- (13.7841, 63.1134) -- (13.5382, 63.0901) -- (13.5382, 63.0901)-- cycle;;



    \node[anchor=south] (text8752) at (38.7675, 39.9415){a};



    \path[draw=black] (12.6372, 64.8286).. controls (12.2548, 65.0011) and (11.7972, 65.2078) .. (11.3915, 65.3909);



    \path[draw=black,fill=black] (11.3425, 65.2776) -- (11.0716, 65.5355) -- (11.4441, 65.5027) -- (11.3425, 65.2776) -- (11.3425, 65.2776)-- cycle;;



    \node[anchor=south] (text2411) at (11.8223, 65.3905){a};



    \path[draw=black] (13.9587, 63.6185).. controls (14.5295, 62.4819) and (15.5515, 60.4464) .. (16.171, 59.2116);



    \path[draw=black,fill=black] (16.2779, 59.2744) -- (16.3255, 58.9037) -- (16.057, 59.1637) -- (16.2779, 59.2744) -- (16.2779, 59.2744)-- cycle;;



    \node[anchor=south] (text4504) at (15.2015, 61.4609){b};



    \path[draw=black] (47.776, 69.4073).. controls (47.5245, 71.1193) and (46.7113, 74.6831) .. (44.3022, 74.6831).. controls (13.4698, 74.6831) and (13.4698, 74.6831) .. (13.4698, 74.6831).. controls (12.4065, 74.6831) and (12.2283, 73.9898) .. (11.7126, 73.0603).. controls (10.6465, 71.138) and (10.3964, 68.531) .. (10.3466, 67.0398);



    \path[draw=black,fill=black] (10.4704, 67.0447) -- (10.3385, 66.6948) -- (10.2235, 67.0504) -- (10.4704, 67.0447) -- (10.4704, 67.0447)-- cycle;;



    \node[anchor=south] (text2658) at (28.3333, 74.8312){c};



    \path[draw=black] (23.8478, 53.0204).. controls (24.3187, 53.6307) and (25.0257, 54.507) .. (25.7126, 55.2097).. controls (26.4834, 55.9982) and (27.0905, 55.849) .. (27.589, 56.8325).. controls (29.1814, 59.9754) and (26.5165, 61.6525) .. (28.224, 64.7347).. controls (28.4533, 65.1485) and (28.6403, 65.1905) .. (29.0781, 65.3697).. controls (35.8334, 68.1351) and (38.2496, 65.9176) .. (45.3461, 67.6275).. controls (45.7884, 67.734) and (46.2626, 67.8879) .. (46.6792, 68.0374);



    \path[draw=black,fill=black] (46.6348, 68.1524) -- (47.0087, 68.1591) -- (46.7205, 67.921) -- (46.6348, 68.1524) -- (46.6348, 68.1524)-- cycle;;



    \node[anchor=south] (text2468) at (35.3882, 67.0056){c};



    \path[draw=black] (24.1593, 52.5932).. controls (24.4556, 52.7085) and (24.7855, 52.8475) .. (25.0776, 52.9943).. controls (25.2187, 53.0652) and (25.3633, 53.1446) .. (25.5055, 53.2264);



    \path[draw=black,fill=black] (25.4363, 53.3291) -- (25.8022, 53.4046) -- (25.5633, 53.1174) -- (25.4363, 53.3291) -- (25.4363, 53.3291)-- cycle;;



    \node[anchor=south] (text2382) at (24.9541, 53.1424){b};



    \path[draw=black] (26.651, 58.1025) ellipse (0.938cm and 0.938cm);



    \node[anchor=south] (text2563) at (26.651, 57.9543){L29};



    \path[draw=black] (23.755, 53.0909).. controls (24.3261, 54.0918) and (25.3023, 55.802) .. (25.9532, 56.9422);



    \path[draw=black,fill=black] (25.8434, 56.999) -- (26.1257, 57.2442) -- (26.0579, 56.8766) -- (25.8434, 56.999) -- (25.8434, 56.999)-- cycle;;



    \node[anchor=south] (text3715) at (24.9541, 55.5149){a};



    \path[draw=black] (45.3073, 55.5353).. controls (45.5274, 55.5985) and (45.7606, 55.6585) .. (45.9811, 55.7036).. controls (52.3039, 56.9951) and (53.9489, 57.3969) .. (60.4023, 57.3969).. controls (60.4023, 57.3969) and (60.4023, 57.3969) .. (67.4744, 57.3969).. controls (68.1831, 57.3969) and (68.9497, 57.5899) .. (69.5646, 57.7956);



    \path[draw=black,fill=black] (69.5205, 57.9109) -- (69.8945, 57.9124) -- (69.6031, 57.6781) -- (69.5205, 57.9109) -- (69.5205, 57.9109)-- cycle;;



    \node[anchor=south] (text5398) at (57.1031, 57.4985){b};



    \path[draw=black] (45.3556, 55.2192).. controls (47.1015, 55.2069) and (50.6853, 55.0598) .. (53.551, 54.1514).. controls (53.7863, 54.077) and (54.0265, 53.9683) .. (54.2488, 53.8515);



    \path[draw=black,fill=black] (54.3091, 53.9595) -- (54.5557, 53.6787) -- (54.1877, 53.7443) -- (54.3091, 53.9595) -- (54.3091, 53.9595)-- cycle;;



    \node[anchor=south] (text5814) at (49.7523, 55.0915){c};



    \path[draw=black] (47.9143, 49.4947) ellipse (0.938cm and 0.938cm);



    \path[draw=black] (47.9143, 49.4947) ellipse (1.0791cm and 1.0791cm);



    \node[anchor=south] (text4457) at (47.9143, 49.3466){L14};



    \path[draw=black] (44.9054, 54.3278).. controls (45.0515, 54.1101) and (45.206, 53.8762) .. (45.3461, 53.6575).. controls (45.9645, 52.6912) and (46.6386, 51.5803) .. (47.1361, 50.7492);



    \path[draw=black,fill=black] (47.2352, 50.8236) -- (47.31, 50.4575) -- (47.0232, 50.697) -- (47.2352, 50.8236) -- (47.2352, 50.8236)-- cycle;;



    \node[anchor=south] (text2828) at (46.0908, 52.7872){a};



    \path[draw=black] (48.732, 48.7775).. controls (49.7173, 47.9213) and (51.5003, 46.5222) .. (53.304, 45.8611).. controls (55.9266, 44.8998) and (59.2413, 44.6807) .. (60.9554, 44.6349);



    \path[draw=black,fill=black] (60.9519, 44.7583) -- (61.3022, 44.6274) -- (60.9469, 44.5114) -- (60.9519, 44.7583) -- (60.9519, 44.7583)-- cycle;;



    \node[anchor=south] (text4569) at (55.2651, 45.7031){c};



    \path[draw=black] (47.081, 48.786).. controls (46.0777, 47.9707) and (44.2676, 46.7431) .. (42.4642, 46.7431).. controls (26.6157, 46.7431) and (26.6157, 46.7431) .. (26.6157, 46.7431).. controls (24.6616, 46.7431) and (23.8284, 49.3088) .. (23.4954, 50.9475);



    \path[draw=black,fill=black] (23.3747, 50.9199) -- (23.4308, 51.2897) -- (23.6174, 50.9658) -- (23.3747, 50.9199) -- (23.3747, 50.9199)-- cycle;;



    \node[anchor=south] (text5695) at (35.3882, 46.8912){a};



    \path[draw=black] (51.5899, 50.2003) ellipse (0.938cm and 0.938cm);



    \path[draw=black] (51.5899, 50.2003) ellipse (1.0791cm and 1.0791cm);



    \node[anchor=south] (text3629) at (51.5899, 50.0521){L27};



    \path[draw=black] (48.9828, 49.6969).. controls (49.3409, 49.7671) and (49.7491, 49.8468) .. (50.1332, 49.9219);



    \path[draw=black,fill=black] (50.099, 50.0412) -- (50.4691, 49.9879) -- (50.1467, 49.7988) -- (50.099, 50.0412) -- (50.099, 50.0412)-- cycle;;



    \node[anchor=south] (text8501) at (49.7523, 50.0168){b};



    \path[draw=black] (61.528, 48.4212).. controls (61.2553, 48.102) and (60.9117, 47.7351) .. (60.561, 47.4486).. controls (58.8726, 46.07) and (58.4013, 45.6657) .. (56.3446, 44.9439).. controls (50.467, 42.8815) and (48.6932, 43.0036) .. (42.4642, 43.0036).. controls (19.8582, 43.0036) and (19.8582, 43.0036) .. (19.8582, 43.0036).. controls (16.9877, 43.0036) and (18.1674, 52.9128) .. (17.4957, 55.7036).. controls (17.3877, 56.1516) and (17.248, 56.6392) .. (17.1196, 57.0618);



    \path[draw=black,fill=black] (17.0035, 57.0191) -- (17.0169, 57.3927) -- (17.2392, 57.0921) -- (17.0035, 57.0191) -- (17.0035, 57.0191)-- cycle;;



    \node[anchor=south] (text4594) at (38.7675, 43.1518){a};



    \path[draw=black] (61.1904, 49.1363).. controls (60.3098, 49.1596) and (58.9509, 49.2732) .. (57.8616, 49.7064).. controls (57.089, 50.0137) and (57.035, 50.3375) .. (56.3446, 50.8).. controls (55.4062, 51.4286) and (55.052, 51.4156) .. (54.186, 52.1406).. controls (53.8625, 52.4115) and (53.9175, 52.6373) .. (53.551, 52.8461).. controls (52.3748, 53.516) and (51.8619, 53.1036) .. (50.5107, 53.0225).. controls (47.2384, 52.826) and (46.403, 52.7791) .. (43.1874, 52.1406).. controls (42.5009, 52.0044) and (42.3531, 51.8724) .. (41.6705, 51.7172).. controls (40.3475, 51.4163) and (40.0142, 51.1835) .. (38.6581, 51.2233).. controls (34.3948, 51.3489) and (33.0137, 50.1438) .. (29.0781, 51.7878).. controls (28.4875, 52.0344) and (27.9467, 52.4912) .. (27.529, 52.9209);



    \path[draw=black,fill=black] (27.4458, 52.8288) -- (27.2969, 53.1717) -- (27.6271, 52.9967) -- (27.4458, 52.8288) -- (27.4458, 52.8288)-- cycle;;



    \node[anchor=south] (text3492) at (44.2669, 52.6863){b};



    \path[draw=black] (65.8798, 54.4714).. controls (66.6037, 54.5264) and (67.5742, 54.6576) .. (68.3772, 54.9698).. controls (69.2951, 55.3268) and (69.9209, 56.2716) .. (70.3012, 57.0604);



    \path[draw=black,fill=black] (70.1841, 57.1013) -- (70.4416, 57.3722) -- (70.4095, 56.9997) -- (70.1841, 57.1013) -- (70.1841, 57.1013)-- cycle;;



    \node[anchor=south] (text176) at (67.4391, 55.118){b};



    \path[draw=black] (63.7039, 54.3006).. controls (61.9157, 54.0738) and (58.2891, 53.6134) .. (56.4402, 53.3788);



    \path[draw=black,fill=black] (56.4607, 53.2571) -- (56.0952, 53.3351) -- (56.4296, 53.5019) -- (56.4607, 53.2571) -- (56.4607, 53.2571)-- cycle;;



    \node[anchor=south] (text6576) at (58.7703, 53.933){c};



    \path[draw=black] (65.4382, 55.3039).. controls (66.348, 56.5626) and (68.1334, 58.9396) .. (69.8941, 60.7483).. controls (70.8914, 61.7728) and (71.1916, 61.9873) .. (72.347, 62.8297).. controls (72.5565, 62.9825) and (72.7876, 63.1363) .. (73.0112, 63.2781);



    \path[draw=black,fill=black] (72.9424, 63.3804) -- (73.3072, 63.4612) -- (73.0723, 63.1705) -- (72.9424, 63.3804) -- (72.9424, 63.3804)-- cycle;;



    \node[anchor=south] (text1624) at (69.1356, 60.168){a};



    \path[draw=black] (77.5324, 63.9233) ellipse (0.9003cm and 0.9003cm);



    \node[anchor=south] (text6735) at (77.5324, 63.7752){q34};



    \path[draw=black] (75.0475, 63.9233).. controls (75.407, 63.9233) and (75.8306, 63.9233) .. (76.2236, 63.9233);



    \path[draw=black,fill=black] (76.212, 64.0468) -- (76.5648, 63.9233) -- (76.212, 63.7999) -- (76.212, 64.0468) -- (76.212, 64.0468)-- cycle;;



    \node[anchor=south] (text8247) at (75.8356, 64.0715){b};



    \path[draw=black] (80.8598, 61.7361) ellipse (0.9003cm and 0.9003cm);



    \node[anchor=south] (text5326) at (80.8598, 61.5879){q35};



    \path[draw=black] (78.2906, 63.4379).. controls (78.7217, 63.1483) and (79.2773, 62.775) .. (79.7585, 62.4522);



    \path[draw=black,fill=black] (79.8241, 62.5567) -- (80.0481, 62.2575) -- (79.6865, 62.3517) -- (79.8241, 62.5567) -- (79.8241, 62.5567)-- cycle;;



    \node[anchor=south] (text8801) at (79.2152, 62.9779){a};



    \path[draw=black] (84.1876, 57.4322) ellipse (0.938cm and 0.938cm);



    \node[anchor=south] (text9575) at (84.1876, 57.2841){L17};



    \path[draw=black] (81.4331, 61.0253).. controls (81.9549, 60.3359) and (82.7528, 59.2815) .. (83.3497, 58.4927);



    \path[draw=black,fill=black] (83.4454, 58.5713) -- (83.5597, 58.2154) -- (83.2481, 58.4221) -- (83.4454, 58.5713) -- (83.4454, 58.5713)-- cycle;;



    \node[anchor=south] (text3294) at (82.5048, 59.8614){c};



    \path[draw=black] (83.2422, 57.2957).. controls (82.5902, 57.2096) and (81.6906, 57.1147) .. (80.8951, 57.1147).. controls (74.1038, 57.1147) and (74.1038, 57.1147) .. (74.1038, 57.1147).. controls (73.3612, 57.1147) and (72.5756, 57.3853) .. (71.9614, 57.6654);



    \path[draw=black,fill=black] (71.9148, 57.5508) -- (71.6509, 57.8157) -- (72.0221, 57.773) -- (71.9148, 57.5508) -- (71.9148, 57.5508)-- cycle;;



    \node[anchor=south] (text9564) at (77.5324, 57.2629){b};



    \path[draw=black] (87.5153, 56.0564) ellipse (0.9003cm and 0.9003cm);



    \node[anchor=south] (text4923) at (87.5153, 55.9082){q46};



    \path[draw=black] (85.0618, 57.078).. controls (85.44, 56.9182) and (85.8929, 56.7267) .. (86.3039, 56.5531);



    \path[draw=black,fill=black] (86.3406, 56.6716) -- (86.6175, 56.4208) -- (86.2443, 56.4444) -- (86.3406, 56.6716) -- (86.3406, 56.6716)-- cycle;;



    \node[anchor=south] (text3037) at (85.8703, 56.8925){c};



    \path[draw=black] (86.6299, 55.8267).. controls (85.9765, 55.6705) and (85.0494, 55.4919) .. (84.2229, 55.4919).. controls (74.1038, 55.4919) and (74.1038, 55.4919) .. (74.1038, 55.4919).. controls (73.0148, 55.4919) and (72.105, 56.4) .. (71.5246, 57.1793);



    \path[draw=black,fill=black] (71.4297, 57.0996) -- (71.3274, 57.459) -- (71.6315, 57.2421) -- (71.4297, 57.0996) -- (71.4297, 57.0996)-- cycle;;



    \node[anchor=south] (text2592) at (79.2152, 55.6401){a};



    \path[draw=black] (27.1872, 54.7088).. controls (27.3311, 54.9522) and (27.4786, 55.2274) .. (27.589, 55.4919).. controls (29.2562, 59.4935) and (25.787, 64.9111) .. (30.1219, 64.9111).. controls (30.1219, 64.9111) and (30.1219, 64.9111) .. (35.4235, 64.9111).. controls (39.7447, 64.9111) and (44.5893, 66.9346) .. (46.7424, 67.9468);



    \path[draw=black,fill=black] (46.6806, 68.054) -- (47.0521, 68.0946) -- (46.7868, 67.8314) -- (46.6806, 68.054) -- (46.6806, 68.054)-- cycle;;



    \node[anchor=south] (text4688) at (37.0847, 65.2688){c};



    \path[draw=black] (30.1572, 53.1989) ellipse (0.938cm and 0.938cm);



    \path[draw=black] (30.1572, 53.1989) ellipse (1.0791cm and 1.0791cm);



    \node[anchor=south] (text984) at (30.1572, 53.0507){L19};



    \path[draw=black] (27.5798, 53.7471).. controls (27.9224, 53.673) and (28.3217, 53.5869) .. (28.7016, 53.5051);



    \path[draw=black,fill=black] (28.7158, 53.6286) -- (29.0343, 53.4335) -- (28.6635, 53.3869) -- (28.7158, 53.6286) -- (28.7158, 53.6286)-- cycle;;



    \node[anchor=south] (text4335) at (28.3333, 53.7397){a};



    \path[draw=black] (29.2033, 52.6807).. controls (28.7355, 52.4461) and (28.1488, 52.1956) .. (27.589, 52.0771).. controls (26.597, 51.8672) and (25.4487, 51.9377) .. (24.589, 52.0499);



    \path[draw=black,fill=black] (24.5763, 51.9268) -- (24.245, 52.0996) -- (24.6119, 52.1712) -- (24.5763, 51.9268) -- (24.5763, 51.9268)-- cycle;;



    \node[anchor=south] (text9844) at (26.651, 52.2252){a};



    \path[draw=black] (31.2364, 53.3485).. controls (33.7136, 53.7034) and (39.9517, 54.5966) .. (42.7881, 55.003);



    \path[draw=black,fill=black] (42.7613, 55.124) -- (43.1281, 55.0517) -- (42.7962, 54.8795) -- (42.7613, 55.124) -- (42.7613, 55.124)-- cycle;;



    \node[anchor=south] (text6391) at (37.0847, 54.4587){b};



    \path[draw=black] (33.6917, 57.5381) ellipse (0.938cm and 0.938cm);



    \node[anchor=south] (text6231) at (33.6917, 57.3899){L20};



    \path[draw=black] (30.8603, 54.0357).. controls (31.4223, 54.7391) and (32.2277, 55.7481) .. (32.8299, 56.5027);



    \path[draw=black,fill=black] (32.7325, 56.5789) -- (33.0493, 56.7775) -- (32.9255, 56.4247) -- (32.7325, 56.5789) -- (32.7325, 56.5789)-- cycle;;



    \node[anchor=south] (text4771) at (31.9952, 55.6468){c};



    \path[draw=black] (34.1461, 58.3727).. controls (34.4216, 58.8712) and (34.8139, 59.4974) .. (35.2647, 59.9722).. controls (36.8561, 61.6476) and (37.3638, 62.1196) .. (39.5122, 62.9708).. controls (46.8372, 65.8731) and (49.1885, 65.4403) .. (57.0678, 65.4403).. controls (57.0678, 65.4403) and (57.0678, 65.4403) .. (64.8219, 65.4403).. controls (67.0415, 65.4403) and (67.8339, 64.849) .. (69.2591, 63.1472).. controls (70.104, 62.1379) and (70.4765, 60.658) .. (70.6402, 59.5947);



    \path[draw=black,fill=black] (70.7619, 59.6149) -- (70.6875, 59.2487) -- (70.5175, 59.5813) -- (70.7619, 59.6149) -- (70.7619, 59.6149)-- cycle;;



    \node[anchor=south] (text3470) at (51.5899, 65.6114){a};



    \path[draw=black] (34.5091, 58.0168).. controls (34.9712, 58.2821) and (35.5762, 58.5996) .. (36.1467, 58.8081).. controls (39.1566, 59.9073) and (40.0173, 59.8932) .. (43.1874, 60.3603).. controls (48.92, 61.2052) and (64.2119, 63.5942) .. (69.2591, 60.7483).. controls (69.7438, 60.4749) and (70.0913, 59.9758) .. (70.3298, 59.4974);



    \path[draw=black,fill=black] (70.4367, 59.5609) -- (70.4695, 59.1887) -- (70.2116, 59.4593) -- (70.4367, 59.5609) -- (70.4367, 59.5609)-- cycle;;



    \node[anchor=south] (text7445) at (51.5899, 61.7548){b};



    \path[draw=black] (37.0847, 57.5381) ellipse (0.938cm and 0.938cm);



    \node[anchor=south] (text4578) at (37.0847, 57.3899){L21};



    \path[draw=black] (34.6435, 57.5381).. controls (34.9814, 57.5381) and (35.3709, 57.5381) .. (35.7367, 57.5381);



    \path[draw=black,fill=black] (35.7297, 57.6615) -- (36.0825, 57.5381) -- (35.7297, 57.4146) -- (35.7297, 57.6615) -- (35.7297, 57.6615)-- cycle;;



    \node[anchor=south] (text15) at (35.3882, 57.6862){c};



    \path[draw=black] (38.0115, 57.7645).. controls (38.9428, 58.0143) and (40.4375, 58.4525) .. (41.6705, 58.9844).. controls (41.9671, 59.1125) and (42.0024, 59.2261) .. (42.3055, 59.3372).. controls (43.9007, 59.9214) and (44.3565, 60.0781) .. (46.0555, 60.0781).. controls (46.0555, 60.0781) and (46.0555, 60.0781) .. (67.4744, 60.0781).. controls (68.3398, 60.0781) and (69.1938, 59.5937) .. (69.8154, 59.1273);



    \path[draw=black,fill=black] (69.8761, 59.237) -- (70.0761, 58.9213) -- (69.7226, 59.0434) -- (69.8761, 59.237) -- (69.8761, 59.237)-- cycle;;



    \node[anchor=south] (text1309) at (53.4275, 60.2262){b};



    \path[draw=black] (37.779, 56.8829).. controls (38.8595, 55.825) and (41.0976, 53.7012) .. (43.1874, 52.1406).. controls (44.0923, 51.465) and (44.5696, 51.6199) .. (45.3461, 50.8).. controls (46.2929, 49.8002) and (45.9673, 49.1525) .. (46.8351, 48.0836).. controls (47.8656, 46.815) and (48.1468, 46.3811) .. (49.6288, 45.6918).. controls (53.4691, 43.9053) and (58.6916, 44.2218) .. (60.9646, 44.4743);



    \path[draw=black,fill=black] (60.948, 44.5968) -- (61.3131, 44.5156) -- (60.9773, 44.3516) -- (60.948, 44.5968) -- (60.948, 44.5968)-- cycle;;



    \node[anchor=south] (text7921) at (49.7523, 45.8399){c};



    \path[draw=black] (40.5913, 61.5597) ellipse (0.938cm and 0.938cm);



    \path[draw=black] (40.5913, 61.5597) ellipse (1.0791cm and 1.0791cm);



    \node[anchor=south] (text6131) at (40.5913, 61.4116){L22};



    \path[draw=black] (37.7186, 58.2387).. controls (38.2411, 58.85) and (39.0084, 59.7482) .. (39.6134, 60.4566);



    \path[draw=black,fill=black] (39.5122, 60.5282) -- (39.8353, 60.7162) -- (39.7002, 60.3677) -- (39.5122, 60.5282) -- (39.5122, 60.5282)-- cycle;;



    \node[anchor=south] (text9041) at (38.7675, 59.6971){a};



    \path[draw=black] (44.2669, 70.0617) ellipse (0.938cm and 0.938cm);



    \node[anchor=south] (text7613) at (44.2669, 69.9135){L23};



    \path[draw=black] (41.0471, 62.5517).. controls (41.7089, 64.1121) and (42.9913, 67.1368) .. (43.7106, 68.8329);



    \path[draw=black,fill=black] (43.5956, 68.8777) -- (43.8468, 69.1543) -- (43.8228, 68.7814) -- (43.5956, 68.8777) -- (43.5956, 68.8777)-- cycle;;



    \node[anchor=south] (text6883) at (42.4289, 66.2139){a};



    \path[draw=black] (44.2669, 57.9967) ellipse (0.938cm and 0.938cm);



    \path[draw=black] (44.2669, 57.9967) ellipse (1.0791cm and 1.0791cm);



    \node[anchor=south] (text3099) at (44.2669, 57.8485){L24};



    \path[draw=black] (41.3734, 60.8214).. controls (41.9015, 60.2996) and (42.6166, 59.593) .. (43.199, 59.0169);



    \path[draw=black,fill=black] (43.271, 59.1196) -- (43.4351, 58.7837) -- (43.0975, 58.9439) -- (43.271, 59.1196) -- (43.271, 59.1196)-- cycle;;



    \node[anchor=south] (text8258) at (42.4289, 60.0333){b};



    \path[draw=black] (41.6719, 61.7336).. controls (46.1913, 62.4685) and (64.0567, 65.1933) .. (69.2591, 63.2531).. controls (70.1587, 62.9176) and (72.0471, 61.1265) .. (73.1823, 59.9948);



    \path[draw=black,fill=black] (73.2681, 60.0837) -- (73.4296, 59.7468) -- (73.0931, 59.9094) -- (73.2681, 60.0837) -- (73.2681, 60.0837)-- cycle;;



    \node[anchor=south] (text8474) at (57.1031, 63.8101){c};



    \path[draw=black] (43.3349, 70.2292).. controls (41.7467, 70.5111) and (38.3318, 71.0494) .. (35.4235, 71.0494).. controls (16.6924, 71.0494) and (16.6924, 71.0494) .. (16.6924, 71.0494).. controls (14.3866, 71.0494) and (13.7657, 67.7273) .. (13.5999, 65.798);



    \path[draw=black,fill=black] (13.7234, 65.7938) -- (13.5742, 65.4509) -- (13.4772, 65.8118) -- (13.7234, 65.7938) -- (13.7234, 65.7938)-- cycle;;



    \node[anchor=south] (text7810) at (28.3333, 71.1976){a};



    \path[draw=black] (45.1386, 69.6951).. controls (45.4074, 69.5777) and (45.7066, 69.4471) .. (45.9811, 69.3279).. controls (46.2142, 69.2266) and (46.464, 69.1187) .. (46.7028, 69.0157);



    \path[draw=black,fill=black] (46.7416, 69.1332) -- (47.0168, 68.8802) -- (46.6439, 68.9063) -- (46.7416, 69.1332) -- (46.7416, 69.1332)-- cycle;;



    \node[anchor=south] (text8342) at (46.0908, 69.4761){c};



    \path[draw=black] (45.2113, 70.201).. controls (45.9327, 70.2998) and (46.9681, 70.4144) .. (47.879, 70.4144).. controls (47.879, 70.4144) and (47.879, 70.4144) .. (74.1743, 70.4144).. controls (76.4861, 70.4144) and (77.1906, 67.1117) .. (77.4044, 65.218);



    \path[draw=black,fill=black] (77.5261, 65.2424) -- (77.4386, 64.879) -- (77.2802, 65.2177) -- (77.5261, 65.2424) -- (77.5261, 65.2424)-- cycle;;



    \node[anchor=south] (text5983) at (60.4375, 70.5626){b};



    \path[draw=black] (45.3083, 58.3195).. controls (46.0188, 58.5191) and (46.9978, 58.7375) .. (47.879, 58.7375).. controls (47.879, 58.7375) and (47.879, 58.7375) .. (53.4628, 58.7375).. controls (56.5471, 58.7375) and (57.3183, 58.7375) .. (60.4023, 58.7375).. controls (60.4023, 58.7375) and (60.4023, 58.7375) .. (67.4744, 58.7375).. controls (68.1496, 58.7375) and (68.8957, 58.6415) .. (69.5064, 58.5378);



    \path[draw=black,fill=black] (69.5279, 58.6592) -- (69.8528, 58.475) -- (69.4838, 58.4165) -- (69.5279, 58.6592) -- (69.5279, 58.6592)-- cycle;;



    \node[anchor=south] (text8714) at (57.1031, 58.8857){b};



    \path[draw=black] (45.0381, 57.2227).. controls (45.4, 56.8304) and (45.8361, 56.3407) .. (46.2001, 55.88).. controls (47.9764, 53.6339) and (48.5708, 53.1372) .. (49.8757, 50.5883).. controls (50.262, 49.8334) and (49.9128, 49.3903) .. (50.5107, 48.7892).. controls (50.7058, 48.5927) and (58.184, 45.9846) .. (61.0217, 44.9996);



    \path[draw=black,fill=black] (61.0623, 45.1164) -- (61.3551, 44.8839) -- (60.9812, 44.8829) -- (61.0623, 45.1164) -- (61.0623, 45.1164)-- cycle;;



    \node[anchor=south] (text627) at (53.4275, 47.8547){c};



    \path[draw=black] (44.9855, 57.1803).. controls (45.116, 57.0025) and (45.2434, 56.8106) .. (45.3461, 56.6208).. controls (46.355, 54.7532) and (47.1001, 52.4023) .. (47.5118, 50.9235);



    \path[draw=black,fill=black] (47.6289, 50.964) -- (47.6028, 50.5912) -- (47.3904, 50.8988) -- (47.6289, 50.964) -- (47.6289, 50.964)-- cycle;;



    \node[anchor=south] (text2420) at (46.0908, 55.2785){a};



    \path[draw=black] (73.1922, 59.037).. controls (72.9174, 59.0169) and (72.6172, 58.9799) .. (72.347, 58.9139).. controls (72.2355, 58.8867) and (72.1219, 58.8525) .. (72.0094, 58.8137);



    \path[draw=black,fill=black] (72.0598, 58.7008) -- (71.6862, 58.6895) -- (71.9713, 58.9315) -- (72.0598, 58.7008) -- (72.0598, 58.7008)-- cycle;;



    \node[anchor=south] (text6357) at (72.4563, 59.1023){a};



    \path[draw=black] (77.5324, 60.1486) ellipse (0.938cm and 0.938cm);



    \node[anchor=south] (text8233) at (77.5324, 60.0004){L26};



    \path[draw=black] (75.0376, 58.747).. controls (75.3364, 58.6856) and (75.6691, 58.6694) .. (75.9591, 58.7798).. controls (76.1958, 58.8698) and (76.4194, 59.0151) .. (76.6191, 59.1778);



    \path[draw=black,fill=black] (76.5274, 59.2614) -- (76.8717, 59.4071) -- (76.6932, 59.0786) -- (76.5274, 59.2614) -- (76.5274, 59.2614)-- cycle;;



    \node[anchor=south] (text9727) at (75.8356, 58.928){c};



    \path[draw=black] (76.6149, 60.3659).. controls (75.7276, 60.5434) and (74.332, 60.7081) .. (73.201, 60.325).. controls (72.6137, 60.126) and (72.0799, 59.6971) .. (71.6689, 59.2829);



    \path[draw=black,fill=black] (71.7603, 59.1996) -- (71.429, 59.0268) -- (71.58, 59.3686) -- (71.7603, 59.1996) -- (71.7603, 59.1996)-- cycle;;



    \node[anchor=south] (text1840) at (74.1391, 60.7064){b};



    \path[draw=black] (76.63, 59.8632).. controls (76.2642, 59.7429) and (75.8317, 59.6004) .. (75.432, 59.4688);



    \path[draw=black,fill=black] (75.4719, 59.352) -- (75.0979, 59.3591) -- (75.3946, 59.5866) -- (75.4719, 59.352) -- (75.4719, 59.352)-- cycle;;



    \node[anchor=south] (text1802) at (75.8356, 59.7828){c};



    \path[draw=black] (51.7881, 51.2745).. controls (52.1127, 52.965) and (53.0094, 56.0564) .. (55.2298, 56.0564).. controls (55.2298, 56.0564) and (55.2298, 56.0564) .. (67.4744, 56.0564).. controls (68.4287, 56.0564) and (69.3134, 56.7055) .. (69.9251, 57.3024);



    \path[draw=black,fill=black] (69.831, 57.3825) -- (70.1654, 57.5497) -- (70.008, 57.2103) -- (69.831, 57.3825) -- (69.831, 57.3825)-- cycle;;



    \node[anchor=south] (text9175) at (60.4375, 56.2046){b};



    \path[draw=black] (52.4432, 50.8885).. controls (53.0109, 51.3666) and (53.7669, 52.003) .. (54.3482, 52.4919);



    \path[draw=black,fill=black] (54.2607, 52.5798) -- (54.61, 52.7124) -- (54.4195, 52.3907) -- (54.2607, 52.5798) -- (54.2607, 52.5798)-- cycle;;



    \node[anchor=south] (text8613) at (53.4275, 51.9564){c};



    \path[draw=black] (55.2651, 49.3889) ellipse (0.938cm and 0.938cm);



    \path[draw=black] (55.2651, 49.3889) ellipse (1.0791cm and 1.0791cm);



    \node[anchor=south] (text791) at (55.2651, 49.2407){L28};



    \path[draw=black] (52.3607, 49.4407).. controls (52.6327, 49.2104) and (52.9597, 48.9871) .. (53.304, 48.8668).. controls (53.4973, 48.7994) and (53.7041, 48.7909) .. (53.9076, 48.8177);



    \path[draw=black,fill=black] (53.8678, 48.9352) -- (54.2392, 48.8946) -- (53.9235, 48.6946) -- (53.8678, 48.9352) -- (53.8678, 48.9352)-- cycle;;



    \node[anchor=south] (text6581) at (53.4275, 49.0149){a};



    \path[draw=black] (54.2689, 48.9335).. controls (54.0378, 48.8386) and (53.7891, 48.7479) .. (53.551, 48.6833).. controls (48.7878, 47.3939) and (47.3989, 48.0836) .. (42.4642, 48.0836).. controls (35.3529, 48.0836) and (35.3529, 48.0836) .. (35.3529, 48.0836).. controls (31.1284, 48.0836) and (26.5109, 50.4098) .. (24.426, 51.599);



    \path[draw=black,fill=black] (24.3681, 51.4897) -- (24.1247, 51.7733) -- (24.4919, 51.7035) -- (24.3681, 51.4897) -- (24.3681, 51.4897)-- cycle;;



    \node[anchor=south] (text6154) at (38.7675, 48.2318){a};



    \path[draw=black] (54.6241, 50.2662).. controls (54.4749, 50.4938) and (54.3197, 50.7414) .. (54.186, 50.9764).. controls (53.8671, 51.537) and (54.0978, 51.9395) .. (53.551, 52.2817).. controls (51.9991, 53.2525) and (47.1745, 51.6319) .. (45.3461, 51.5479).. controls (44.3876, 51.5038) and (44.1448, 51.4847) .. (43.1874, 51.5479).. controls (41.5435, 51.6562) and (40.9317, 51.1983) .. (39.5122, 52.0347).. controls (38.554, 52.5992) and (38.7914, 53.2416) .. (38.0231, 54.0456).. controls (37.0219, 55.093) and (35.7099, 56.1178) .. (34.8008, 56.7835);



    \path[draw=black,fill=black] (34.7299, 56.6822) -- (34.5165, 56.9891) -- (34.8749, 56.8822) -- (34.7299, 56.6822) -- (34.7299, 56.6822)-- cycle;;



    \node[anchor=south] (text1252) at (44.2669, 51.6961){c};



    \path[draw=black] (54.1969, 49.6214).. controls (53.8385, 49.7022) and (53.4303, 49.7939) .. (53.0465, 49.8803);



    \path[draw=black,fill=black] (53.0271, 49.7582) -- (52.7103, 49.9562) -- (53.0814, 49.9992) -- (53.0271, 49.7582) -- (53.0271, 49.7582)-- cycle;;



    \node[anchor=south] (text2385) at (53.4275, 49.9671){b};



    \path[draw=black] (25.697, 58.1092).. controls (23.8901, 58.1219) and (19.8766, 58.1508) .. (17.9035, 58.1649);



    \path[draw=black,fill=black] (17.9088, 58.0415) -- (17.557, 58.1674) -- (17.9105, 58.2884) -- (17.9088, 58.0415) -- (17.9088, 58.0415)-- cycle;;



    \node[anchor=south] (text3703) at (21.5611, 58.2877){b};



    \path[draw=black] (25.9158, 57.4933).. controls (24.7149, 56.448) and (22.2603, 54.3112) .. (20.8964, 53.1241);



    \path[draw=black,fill=black] (20.9942, 53.0454) -- (20.647, 52.9068) -- (20.8322, 53.2317) -- (20.9942, 53.0454) -- (20.9942, 53.0454)-- cycle;;



    \node[anchor=south] (text9396) at (23.2576, 56.1082){a};



    \path[draw=black] (26.6559, 59.0465).. controls (26.6457, 60.6845) and (26.8125, 64.1103) .. (28.224, 66.5339).. controls (28.7775, 67.4843) and (29.0223, 68.1214) .. (30.1219, 68.1214).. controls (30.1219, 68.1214) and (30.1219, 68.1214) .. (44.3022, 68.1214).. controls (45.0727, 68.1214) and (45.9317, 68.2117) .. (46.6146, 68.3048);



    \path[draw=black,fill=black] (46.5949, 68.4269) -- (46.9618, 68.3549) -- (46.6302, 68.1824) -- (46.5949, 68.4269) -- (46.5949, 68.4269)-- cycle;;



    \node[anchor=south] (text9035) at (37.0847, 68.2696){c};



  \end{scope}

\end{tikzpicture}

    \end{adjustbox}
    \caption{Минимальный ДКА}
    \label{fig:dfa}
\end{figure}

Таблица классов эквивалентности:

\begin{table}[h]
\caption{Таблица эквивалентности (строки - префиксы, столбцы - суффиксы)}
\centering
\begin{tabular}{l|*{14}{c|}}
\cline{2-15}
 & $\varepsilon$ & aaa & aba & cc & ba & a & b & bacb & c & acb & cb & aa & caba & ccb \\ \hline
\multicolumn{1}{|l|}{$\varepsilon$} & + & - & + & - & - & - & + & - & - & - & - & - & - & + \\ \hline
\multicolumn{1}{|l|}{a} & - & - & - & - & + & - & - & + & - & - & - & - & - & - \\ \hline
\multicolumn{1}{|l|}{b} & + & - & - & - & - & - & + & - & - & - & - & - & - & + \\ \hline
\multicolumn{1}{|l|}{c} & - & - & - & - & - & + & - & - & - & - & + & - & + & - \\ \hline
\multicolumn{1}{|l|}{aa} & - & - & + & - & - & - & - & - & - & - & - & - & - & - \\ \hline
\multicolumn{1}{|l|}{ab} & - & - & + & - & - & + & - & - & - & + & - & - & - & - \\ \hline
\multicolumn{1}{|l|}{ac} & - & - & - & - & - & - & - & - & - & - & - & - & - & - \\ \hline
\multicolumn{1}{|l|}{ba} & - & - & - & - & - & - & - & + & - & - & - & - & - & - \\ \hline
\multicolumn{1}{|l|}{bc} & - & - & - & - & - & + & - & - & - & - & + & - & - & - \\ \hline
\multicolumn{1}{|l|}{cc} & - & - & + & - & - & - & + & - & - & - & - & - & - & + \\ \hline
\multicolumn{1}{|l|}{aaa} & - & - & - & - & + & - & - & - & - & - & - & - & - & - \\ \hline
\multicolumn{1}{|l|}{aac} & - & - & - & - & - & - & - & - & - & - & - & - & + & - \\ \hline
\multicolumn{1}{|l|}{aba} & + & + & + & - & + & - & + & - & - & - & + & - & - & + \\ \hline
\multicolumn{1}{|l|}{bab} & - & - & - & - & - & - & - & - & - & + & - & - & - & - \\ \hline
\multicolumn{1}{|l|}{bcc} & - & - & - & - & - & - & + & - & - & - & - & - & - & + \\ \hline
\multicolumn{1}{|l|}{aaab} & - & - & + & - & - & + & - & - & - & - & - & - & - & - \\ \hline
\multicolumn{1}{|l|}{abaa} & - & - & + & - & + & - & - & + & - & - & - & + & - & - \\ \hline
\multicolumn{1}{|l|}{abab} & + & - & + & + & - & + & + & - & - & - & - & - & - & + \\ \hline
\multicolumn{1}{|l|}{abac} & - & - & - & - & - & + & + & - & - & - & + & - & + & - \\ \hline
\multicolumn{1}{|l|}{baba} & - & - & - & - & - & - & - & - & - & - & + & - & - & - \\ \hline
\multicolumn{1}{|l|}{aaaba} & + & + & + & - & + & - & + & - & - & - & - & - & - & + \\ \hline
\multicolumn{1}{|l|}{abaaa} & - & - & + & - & + & + & - & - & - & - & - & - & - & - \\ \hline
\multicolumn{1}{|l|}{ababa} & + & + & + & - & + & - & + & + & - & - & - & - & - & + \\ \hline
\multicolumn{1}{|l|}{ababc} & - & - & - & - & - & + & - & - & + & - & + & - & + & - \\ \hline
\multicolumn{1}{|l|}{abacc} & - & - & + & - & - & + & + & - & - & - & - & - & - & + \\ \hline
\multicolumn{1}{|l|}{babac} & - & - & - & - & - & - & + & - & - & - & - & - & - & - \\ \hline
\multicolumn{1}{|l|}{ababab} & + & - & + & + & - & + & + & - & - & + & - & - & - & + \\ \hline
\multicolumn{1}{|l|}{ababcc} & + & + & + & - & - & - & + & - & - & - & - & - & - & + \\ \hline
\multicolumn{1}{|l|}{abacca} & + & - & - & - & + & - & + & - & - & - & - & - & - & + \\ \hline
\multicolumn{1}{|l|}{babacc} & - & - & - & - & - & + & - & - & - & - & - & - & - & - \\ \hline
\multicolumn{1}{|l|}{abababa} & + & + & + & - & + & - & + & + & - & - & + & - & - & + \\ \hline
\multicolumn{1}{|l|}{ababcca} & - & - & - & - & + & - & - & + & - & - & - & + & - & - \\ \hline
\multicolumn{1}{|l|}{ababccb} & + & - & - & + & - & - & + & - & - & - & - & - & - & + \\ \hline
\multicolumn{1}{|l|}{abaccaa} & - & - & + & - & - & - & - & + & - & - & - & - & - & - \\ \hline
\multicolumn{1}{|l|}{abaccab} & + & - & + & - & - & + & + & - & - & - & - & - & - & + \\ \hline
\end{tabular}
\end{table}

\clearpage
\newpage

\section{Меньший НКА}

\begin{figure}[H]
    \centering
    \begin{adjustbox}{width=\textwidth}
    \def \globalscale {1.000000}
\begin{tikzpicture}[y=1cm, x=1cm, yscale=\globalscale,xscale=\globalscale, every node/.append style={scale=\globalscale}, inner sep=0pt, outer sep=0pt]
  \begin{scope}[shift={(0.1411, -27.3685)}]
    \path[fill=white] (-0.1411, 27.3756) -- (-0.1411, 54.8852) -- (35.2647, 54.8852) -- (35.2647, 27.3756) -- (-0.1411, 27.3756) -- (-0.1411, 27.3756)-- cycle;;



    \path[draw=black] (2.2084, 40.8517) ellipse (0.635cm and 0.635cm);



    \path[draw=black] (2.2084, 40.8517) ellipse (0.7761cm and 0.7761cm);



    \node[anchor=south] (text3898) at (2.2084, 40.7035){0};



    \path[draw=black] (19.7637, 32.4908) ellipse (0.635cm and 0.635cm);



    \path[draw=black] (19.7637, 32.4908) ellipse (0.7761cm and 0.7761cm);



    \node[anchor=south] (text9468) at (19.7637, 32.3427){5};



    \path[draw=black] (2.34, 40.0717).. controls (2.6166, 38.0079) and (3.4957, 32.5614) .. (5.0733, 32.5614).. controls (5.0733, 32.5614) and (5.0733, 32.5614) .. (16.6349, 32.5614).. controls (17.2787, 32.5614) and (17.9984, 32.5459) .. (18.5812, 32.5293);



    \path[draw=black,fill=black] (18.5762, 32.6531) -- (18.9251, 32.5191) -- (18.5685, 32.4062) -- (18.5762, 32.6531) -- (18.5762, 32.6531)-- cycle;;



    \node[anchor=south] (text7140) at (10.6271, 32.7096){b};



    \path[draw=black] (5.1086, 43.3564) ellipse (0.635cm and 0.635cm);



    \node[anchor=south] (text1467) at (5.1086, 43.2082){1};



    \path[draw=black] (2.8162, 41.3586).. controls (3.2466, 41.7396) and (3.8407, 42.2652) .. (4.3113, 42.6815);



    \path[draw=black,fill=black] (4.2277, 42.7725) -- (4.5738, 42.914) -- (4.3914, 42.5877) -- (4.2277, 42.7725) -- (4.2277, 42.7725)-- cycle;;



    \node[anchor=south] (text871) at (3.7292, 42.4021){a};



    \path[draw=black] (19.7637, 46.9194) ellipse (0.635cm and 0.635cm);



    \node[anchor=south] (text6740) at (19.7637, 46.7713){2};



    \path[draw=black] (2.2691, 41.6373).. controls (2.3735, 44.2196) and (2.8758, 52.2817) .. (5.0733, 52.2817).. controls (5.0733, 52.2817) and (5.0733, 52.2817) .. (9.2826, 52.2817).. controls (11.6318, 52.2817) and (12.1373, 51.496) .. (14.2392, 50.4472).. controls (15.9875, 49.5748) and (17.8721, 48.2762) .. (18.9156, 47.522);



    \path[draw=black,fill=black] (18.9801, 47.6278) -- (19.1925, 47.3202) -- (18.8348, 47.4282) -- (18.9801, 47.6278) -- (18.9801, 47.6278)-- cycle;;



    \node[anchor=south] (text4725) at (10.6271, 52.4002){a};



    \path[draw=black] (22.7591, 30.1978) ellipse (0.635cm and 0.635cm);



    \node[anchor=south] (text9612) at (22.7591, 30.0496){3};



    \path[draw=black] (2.2602, 40.0646).. controls (2.3368, 37.3962) and (2.7626, 28.8572) .. (5.0733, 28.8572).. controls (5.0733, 28.8572) and (5.0733, 28.8572) .. (19.7989, 28.8572).. controls (20.5595, 28.8572) and (21.3385, 29.2365) .. (21.9012, 29.5917);



    \path[draw=black,fill=black] (21.8313, 29.6933) -- (22.1929, 29.7875) -- (21.9689, 29.4883) -- (21.8313, 29.6933) -- (21.8313, 29.6933)-- cycle;;



    \node[anchor=south] (text9653) at (12.0206, 29.0054){a};



    \path[draw=black] (10.6271, 39.7933) ellipse (0.635cm and 0.635cm);



    \node[anchor=south] (text1357) at (10.6271, 39.6452){4};



    \path[draw=black] (2.9845, 40.7578).. controls (4.4856, 40.5673) and (7.8867, 40.1362) .. (9.586, 39.9207);



    \path[draw=black,fill=black] (9.5917, 40.0445) -- (9.9265, 39.8776) -- (9.5606, 39.7997) -- (9.5917, 40.0445) -- (9.5917, 40.0445)-- cycle;;



    \node[anchor=south] (text4573) at (6.4883, 40.4707){a};



    \path[draw=black] (19.7637, 51.8936) ellipse (0.635cm and 0.635cm);



    \node[anchor=south] (text9669) at (19.7637, 51.7454){6};



    \path[draw=black] (2.2423, 41.6437).. controls (2.2574, 44.5072) and (2.5001, 54.1514) .. (5.0733, 54.1514).. controls (5.0733, 54.1514) and (5.0733, 54.1514) .. (16.6349, 54.1514).. controls (17.6414, 54.1514) and (18.5395, 53.3615) .. (19.1054, 52.7163);



    \path[draw=black,fill=black] (19.1985, 52.7974) -- (19.3287, 52.4471) -- (19.0084, 52.6397) -- (19.1985, 52.7974) -- (19.1985, 52.7974)-- cycle;;



    \node[anchor=south] (text6428) at (10.6271, 54.2996){c};



    \path[draw=black] (22.7591, 40.8164) ellipse (0.635cm and 0.635cm);



    \node[anchor=south] (text1855) at (22.7591, 40.6682){7};



    \path[draw=black] (2.9284, 41.184).. controls (4.1455, 41.758) and (6.7797, 42.9246) .. (9.138, 43.4622).. controls (13.1449, 44.3759) and (14.4671, 45.5782) .. (18.3526, 44.2383).. controls (19.8713, 43.7145) and (21.2658, 42.4219) .. (22.0536, 41.5837);



    \path[draw=black,fill=black] (22.1417, 41.6705) -- (22.2889, 41.3269) -- (21.9597, 41.5036) -- (22.1417, 41.6705) -- (22.1417, 41.6705)-- cycle;;



    \node[anchor=south] (text4722) at (12.0206, 44.4257){c};



    \path[draw=black] (34.4886, 34.5017) ellipse (0.635cm and 0.635cm);



    \node[anchor=south] (text4028) at (34.4886, 34.3535){8};



    \path[draw=black] (2.2416, 40.0674).. controls (2.255, 37.2089) and (2.491, 27.5167) .. (5.0733, 27.5167).. controls (5.0733, 27.5167) and (5.0733, 27.5167) .. (31.6699, 27.5167).. controls (34.194, 27.5167) and (34.4932, 31.5024) .. (34.4883, 33.4522);



    \path[draw=black,fill=black] (34.3648, 33.4493) -- (34.4837, 33.8035) -- (34.6117, 33.4525) -- (34.3648, 33.4493) -- (34.3648, 33.4493)-- cycle;;



    \node[anchor=south] (text1123) at (18.2291, 27.6648){c};



    \path[draw=black] (19.4853, 33.2313).. controls (19.4511, 33.5876) and (19.5439, 33.9019) .. (19.7637, 33.9019).. controls (19.8942, 33.9019) and (19.9799, 33.7912) .. (20.0208, 33.6264);



    \path[draw=black,fill=black] (20.1436, 33.6434) -- (20.0388, 33.2846) -- (19.897, 33.6303) -- (20.1436, 33.6434) -- (20.1436, 33.6434)-- cycle;;



    \node[anchor=south] (text2101) at (19.7637, 34.0501){b};



    \path[draw=black] (20.3874, 32.0248).. controls (20.6308, 31.8322) and (20.9162, 31.6071) .. (21.1748, 31.4043).. controls (21.4182, 31.2134) and (21.6835, 31.0063) .. (21.9248, 30.8183);



    \path[draw=black,fill=black] (21.9901, 30.9241) -- (22.1925, 30.6102) -- (21.8384, 30.7294) -- (21.9901, 30.9241) -- (21.9901, 30.9241)-- cycle;;



    \node[anchor=south] (text5117) at (21.2845, 31.5524){a};



    \path[draw=black] (20.0427, 33.2331).. controls (20.1891, 33.6815) and (20.3775, 34.2646) .. (20.5398, 34.7839).. controls (20.9314, 36.0377) and (21.0259, 36.3523) .. (21.3939, 37.6132).. controls (21.6853, 38.6108) and (21.5745, 38.929) .. (22.0289, 39.8639).. controls (22.0511, 39.9094) and (22.0754, 39.9553) .. (22.1015, 40.0004);



    \path[draw=black,fill=black] (21.9943, 40.0622) -- (22.2913, 40.289) -- (22.2003, 39.9263) -- (21.9943, 40.0622) -- (21.9943, 40.0622)-- cycle;;



    \node[anchor=south] (text2090) at (21.2845, 37.7613){c};



    \path[draw=black] (20.151, 33.1802).. controls (20.387, 33.5696) and (20.7349, 34.0293) .. (21.1748, 34.29).. controls (23.4961, 35.6651) and (24.4775, 35.2072) .. (27.1752, 35.2072).. controls (27.1752, 35.2072) and (27.1752, 35.2072) .. (31.6699, 35.2072).. controls (32.2975, 35.2072) and (32.9812, 35.0365) .. (33.5139, 34.8633);



    \path[draw=black,fill=black] (33.5453, 34.9829) -- (33.8377, 34.7504) -- (33.4641, 34.7497) -- (33.5453, 34.9829) -- (33.5453, 34.9829)-- cycle;;



    \node[anchor=south] (text7731) at (27.2105, 35.3554){c};



    \path[draw=black] (16.5996, 39.9344) ellipse (0.7299cm and 0.7299cm);



    \path[draw=black] (16.5996, 39.9344) ellipse (0.871cm and 0.871cm);



    \node[anchor=south] (text6697) at (16.5996, 39.7863){14};



    \path[draw=black] (16.9658, 39.1379).. controls (17.5235, 37.7959) and (18.651, 35.0827) .. (19.2797, 33.5707);



    \path[draw=black,fill=black] (19.3897, 33.6271) -- (19.4112, 33.2542) -- (19.1618, 33.5326) -- (19.3897, 33.6271) -- (19.3897, 33.6271)-- cycle;;



    \node[anchor=south] (text8633) at (18.2291, 36.3608){b};



    \path[draw=black] (15.7554, 40.1884).. controls (15.5406, 40.2558) and (15.3081, 40.3274) .. (15.0932, 40.3931).. controls (11.8054, 41.3953) and (7.8948, 42.5489) .. (6.1136, 43.0721);



    \path[draw=black,fill=black] (6.0829, 42.9525) -- (5.7792, 43.1701) -- (6.1524, 43.1895) -- (6.0829, 42.9525) -- (6.0829, 42.9525)-- cycle;;



    \node[anchor=south] (text7650) at (10.6271, 42.0529){a};



    \path[draw=black] (16.9834, 40.7222).. controls (17.5574, 42.0172) and (18.6972, 44.5915) .. (19.3121, 45.9796);



    \path[draw=black,fill=black] (19.1968, 46.0244) -- (19.4525, 46.2968) -- (19.4225, 45.9243) -- (19.1968, 46.0244) -- (19.1968, 46.0244)-- cycle;;



    \node[anchor=south] (text1560) at (18.2291, 43.7847){a};



    \path[draw=black] (16.9933, 39.1495).. controls (17.1552, 38.7738) and (17.3394, 38.3138) .. (17.4706, 37.8883).. controls (18.4203, 34.8118) and (16.6546, 33.0235) .. (18.9876, 30.8046).. controls (19.7051, 30.1219) and (20.8728, 30.0327) .. (21.7121, 30.075);



    \path[draw=black,fill=black] (21.7022, 30.1981) -- (22.0634, 30.1022) -- (21.7212, 29.9519) -- (21.7022, 30.1981) -- (21.7022, 30.1981)-- cycle;;



    \node[anchor=south] (text8248) at (19.7637, 30.9527){a};



    \path[draw=black] (15.712, 39.914).. controls (14.6424, 39.8886) and (12.8097, 39.8448) .. (11.6713, 39.8173);



    \path[draw=black,fill=black] (11.6808, 39.6942) -- (11.3252, 39.8092) -- (11.6748, 39.9411) -- (11.6808, 39.6942) -- (11.6808, 39.6942)-- cycle;;



    \node[anchor=south] (text5765) at (13.5093, 40.0262){a};



    \path[draw=black] (16.8547, 40.785).. controls (17.411, 42.9352) and (18.8715, 48.5807) .. (19.4684, 50.8885);



    \path[draw=black,fill=black] (19.3481, 50.9168) -- (19.5559, 51.2276) -- (19.5873, 50.855) -- (19.3481, 50.9168) -- (19.3481, 50.9168)-- cycle;;



    \node[anchor=south] (text9234) at (18.2291, 46.4206){c};



    \path[draw=black] (17.4417, 40.2135).. controls (17.6565, 40.2858) and (17.8894, 40.3617) .. (18.1056, 40.4283).. controls (18.4947, 40.5479) and (18.5854, 40.6121) .. (18.9876, 40.6753).. controls (19.903, 40.8189) and (20.9596, 40.846) .. (21.7149, 40.8421);



    \path[draw=black,fill=black] (21.7068, 40.9656) -- (22.0578, 40.8379) -- (21.7036, 40.7187) -- (21.7068, 40.9656) -- (21.7068, 40.9656)-- cycle;;



    \node[anchor=south] (text9758) at (19.7637, 40.9674){c};



    \path[draw=black] (17.2053, 40.5673).. controls (18.2104, 41.6084) and (20.4093, 43.5681) .. (22.7238, 43.5681).. controls (22.7238, 43.5681) and (22.7238, 43.5681) .. (31.6699, 43.5681).. controls (33.3357, 43.5681) and (34.1073, 37.9035) .. (34.3574, 35.5293);



    \path[draw=black,fill=black] (34.4787, 35.5558) -- (34.3916, 35.1924) -- (34.2332, 35.5307) -- (34.4787, 35.5558) -- (34.4787, 35.5558)-- cycle;;



    \node[anchor=south] (text1826) at (25.7358, 43.7162){c};



    \path[draw=black] (19.7637, 35.8422) ellipse (0.7299cm and 0.7299cm);



    \node[anchor=south] (text4664) at (19.7637, 35.6941){16};



    \path[draw=black] (17.1503, 39.2409).. controls (17.4315, 38.8631) and (17.7867, 38.3879) .. (18.1056, 37.9659).. controls (18.415, 37.5564) and (18.76, 37.1045) .. (19.0521, 36.7231);



    \path[draw=black,fill=black] (19.1438, 36.8064) -- (19.2603, 36.4511) -- (18.9477, 36.6561) -- (19.1438, 36.8064) -- (19.1438, 36.8064)-- cycle;;



    \node[anchor=south] (text4481) at (18.2291, 38.1141){a};



    \path[draw=black] (19.7637, 37.9236) ellipse (0.7299cm and 0.7299cm);



    \node[anchor=south] (text9963) at (19.7637, 37.7754){17};



    \path[draw=black] (17.3112, 39.4088).. controls (17.557, 39.2247) and (17.8396, 39.0211) .. (18.1056, 38.8479).. controls (18.3109, 38.7142) and (18.5367, 38.5787) .. (18.7526, 38.4545);



    \path[draw=black,fill=black] (18.8083, 38.565) -- (19.0549, 38.2842) -- (18.687, 38.3498) -- (18.8083, 38.565) -- (18.8083, 38.565)-- cycle;;



    \node[anchor=south] (text5283) at (18.2291, 38.9961){b};



    \path[draw=black,fill=black] (0.0635, 40.8517) ellipse (0.0635cm and 0.0635cm);



    \path[draw=black] (0.1348, 40.8517).. controls (0.2501, 40.8517) and (0.6265, 40.8517) .. (1.0252, 40.8517);



    \path[draw=black,fill=black] (1.016, 40.9751) -- (1.3688, 40.8517) -- (1.016, 40.7282) -- (1.016, 40.9751) -- (1.016, 40.9751)-- cycle;;



    \path[draw=black] (7.868, 45.6142) ellipse (0.635cm and 0.635cm);



    \node[anchor=south] (text3697) at (7.868, 45.466){9};



    \path[draw=black] (5.7358, 43.4883).. controls (6.0166, 43.5712) and (6.3454, 43.7003) .. (6.598, 43.8926).. controls (6.8848, 44.1113) and (7.1335, 44.4151) .. (7.3307, 44.7068);



    \path[draw=black,fill=black] (7.2249, 44.771) -- (7.5166, 45.0049) -- (7.4344, 44.6401) -- (7.2249, 44.771) -- (7.2249, 44.771)-- cycle;;



    \node[anchor=south] (text5958) at (6.4883, 44.0408){a};



    \path[draw=black] (19.2303, 46.5628).. controls (18.6404, 46.1761) and (17.6142, 45.6142) .. (16.6349, 45.6142).. controls (10.5918, 45.6142) and (10.5918, 45.6142) .. (10.5918, 45.6142).. controls (10.0341, 45.6142) and (9.4107, 45.6142) .. (8.9052, 45.6142);



    \path[draw=black,fill=black] (8.9179, 45.4907) -- (8.5651, 45.6142) -- (8.9179, 45.7376) -- (8.9179, 45.4907) -- (8.9179, 45.4907)-- cycle;;



    \node[anchor=south] (text2516) at (13.5093, 45.7623){b};



    \path[draw=black] (25.7358, 30.1978) ellipse (0.7299cm and 0.7299cm);



    \node[anchor=south] (text9526) at (25.7358, 30.0496){10};



    \path[draw=black] (23.4107, 30.1978).. controls (23.7543, 30.1978) and (24.1931, 30.1978) .. (24.5943, 30.1978);



    \path[draw=black,fill=black] (24.5868, 30.3213) -- (24.9396, 30.1978) -- (24.5868, 30.0743) -- (24.5868, 30.3213) -- (24.5868, 30.3213)-- cycle;;



    \node[anchor=south] (text4640) at (24.2475, 30.3459){b};



    \path[draw=black] (13.5093, 37.9236) ellipse (0.7299cm and 0.7299cm);



    \node[anchor=south] (text6489) at (13.5093, 37.7754){11};



    \path[draw=black] (11.145, 39.4017).. controls (11.369, 39.2257) and (11.6417, 39.0197) .. (11.8971, 38.8479).. controls (12.0879, 38.7195) and (12.2971, 38.5897) .. (12.4993, 38.4697);



    \path[draw=black,fill=black] (12.5599, 38.5773) -- (12.803, 38.293) -- (12.4358, 38.3635) -- (12.5599, 38.5773) -- (12.5599, 38.5773)-- cycle;;



    \node[anchor=south] (text5260) at (12.0206, 38.9961){b};



    \path[draw=black] (19.1484, 51.683).. controls (17.6251, 51.1249) and (13.3936, 49.5402) .. (9.9921, 47.9425).. controls (9.6023, 47.7594) and (9.4562, 47.7746) .. (9.138, 47.4839).. controls (8.8385, 47.2105) and (8.5778, 46.8496) .. (8.3746, 46.5176);



    \path[draw=black,fill=black] (8.4868, 46.4651) -- (8.2035, 46.2209) -- (8.2726, 46.5882) -- (8.4868, 46.4651) -- (8.4868, 46.4651)-- cycle;;



    \node[anchor=south] (text2721) at (13.5093, 49.9142){c};



    \path[draw=black] (25.7358, 38.1706) ellipse (0.7299cm and 0.7299cm);



    \node[anchor=south] (text497) at (25.7358, 38.0224){12};



    \path[draw=black] (23.2484, 40.4029).. controls (23.6834, 40.0068) and (24.3501, 39.4) .. (24.8786, 38.9188);



    \path[draw=black,fill=black] (24.9537, 39.0176) -- (25.1315, 38.6888) -- (24.7876, 38.8348) -- (24.9537, 39.0176) -- (24.9537, 39.0176)-- cycle;;



    \node[anchor=south] (text6590) at (24.2475, 39.721){c};



    \path[draw=black] (33.8882, 34.2706).. controls (33.3333, 34.0649) and (32.4573, 33.7961) .. (31.6699, 33.7961).. controls (22.7238, 33.7961) and (22.7238, 33.7961) .. (22.7238, 33.7961).. controls (22.0169, 33.7961) and (21.2905, 33.474) .. (20.7387, 33.1498);



    \path[draw=black,fill=black] (20.8128, 33.0507) -- (20.4484, 32.9685) -- (20.682, 33.2602) -- (20.8128, 33.0507) -- (20.8128, 33.0507)-- cycle;;



    \node[anchor=south] (text1897) at (27.2105, 33.9443){a};



    \path[draw=black] (7.3402, 45.2476).. controls (7.0549, 45.031) and (6.6915, 44.7488) .. (6.3786, 44.4853).. controls (6.2163, 44.3484) and (6.0459, 44.1974) .. (5.8854, 44.0521);



    \path[draw=black,fill=black] (5.9761, 43.9678) -- (5.6328, 43.8196) -- (5.8088, 44.1494) -- (5.9761, 43.9678) -- (5.9761, 43.9678)-- cycle;;



    \node[anchor=south] (text7713) at (6.4883, 44.7933){a};



    \path[draw=black] (8.2825, 46.1056).. controls (8.7567, 46.6594) and (9.6252, 47.4839) .. (10.5918, 47.4839).. controls (10.5918, 47.4839) and (10.5918, 47.4839) .. (16.6349, 47.4839).. controls (17.3535, 47.4839) and (18.1508, 47.3347) .. (18.7561, 47.1897);



    \path[draw=black,fill=black] (18.778, 47.3114) -- (19.0895, 47.105) -- (18.717, 47.0722) -- (18.778, 47.3114) -- (18.778, 47.3114)-- cycle;;



    \node[anchor=south] (text7044) at (13.5093, 47.6321){a};



    \path[draw=black] (8.1478, 45.0303).. controls (8.5252, 44.1371) and (9.2809, 42.381) .. (9.9921, 40.9222).. controls (10.0288, 40.8474) and (10.0672, 40.7702) .. (10.1067, 40.6933);



    \path[draw=black,fill=black] (10.2122, 40.7578) -- (10.2665, 40.3878) -- (9.9935, 40.6432) -- (10.2122, 40.7578) -- (10.2122, 40.7578)-- cycle;;



    \node[anchor=south] (text695) at (9.2474, 42.8614){a};



    \path[draw=black] (8.1185, 46.2072).. controls (8.3266, 46.7279) and (8.678, 47.4881) .. (9.138, 48.0483).. controls (9.8203, 48.8795) and (12.0131, 50.1537) .. (12.779, 50.4825).. controls (15.0223, 51.4452) and (15.6901, 51.4703) .. (18.1056, 51.8231).. controls (18.3042, 51.852) and (18.5184, 51.8703) .. (18.7233, 51.8816);



    \path[draw=black,fill=black] (18.7142, 52.0051) -- (19.0712, 51.8943) -- (18.723, 51.7581) -- (18.7142, 52.0051) -- (18.7142, 52.0051)-- cycle;;



    \node[anchor=south] (text8443) at (13.5093, 51.2018){c};



    \path[draw=black] (28.6851, 30.3389) ellipse (0.7299cm and 0.7299cm);



    \node[anchor=south] (text5973) at (28.6851, 30.1907){13};



    \path[draw=black] (26.4682, 30.232).. controls (26.7885, 30.2475) and (27.1769, 30.2666) .. (27.5364, 30.2842);



    \path[draw=black,fill=black] (27.5276, 30.4073) -- (27.886, 30.3015) -- (27.5399, 30.1607) -- (27.5276, 30.4073) -- (27.5276, 30.4073)-- cycle;;



    \node[anchor=south] (text1951) at (27.2105, 30.4207){a};



    \path[draw=black] (14.1563, 38.27).. controls (14.4445, 38.438) and (14.7909, 38.6468) .. (15.0932, 38.8479).. controls (15.246, 38.9495) and (15.4051, 39.0606) .. (15.5593, 39.1721);



    \path[draw=black,fill=black] (15.4852, 39.2709) -- (15.8425, 39.3802) -- (15.6316, 39.0719) -- (15.4852, 39.2709) -- (15.4852, 39.2709)-- cycle;;



    \node[anchor=south] (text883) at (14.9839, 38.9961){a};



    \path[draw=black] (25.05, 37.8749).. controls (23.9402, 37.3599) and (21.7452, 36.2941) .. (21.1748, 35.6658).. controls (20.653, 35.0912) and (20.3077, 34.2879) .. (20.0932, 33.6331);



    \path[draw=black,fill=black] (20.2124, 33.601) -- (19.9919, 33.299) -- (19.976, 33.6726) -- (20.2124, 33.601) -- (20.2124, 33.601)-- cycle;;



    \node[anchor=south] (text7268) at (22.7591, 37.2445){b};



    \path[draw=black] (24.9999, 38.2722).. controls (24.704, 38.3441) and (24.3713, 38.4655) .. (24.124, 38.6715).. controls (23.6626, 39.0557) and (23.8079, 39.3552) .. (23.489, 39.8639).. controls (23.4622, 39.9069) and (23.434, 39.9503) .. (23.405, 39.9941);



    \path[draw=black,fill=black] (23.3077, 39.9179) -- (23.2061, 40.2777) -- (23.5098, 40.0597) -- (23.3077, 39.9179) -- (23.3077, 39.9179)-- cycle;;



    \node[anchor=south] (text972) at (24.2475, 38.8197){c};



    \path[draw=black] (26.4728, 38.0474).. controls (27.8476, 37.7836) and (30.9598, 37.0724) .. (33.2186, 35.7011).. controls (33.4165, 35.5808) and (33.6081, 35.4235) .. (33.7785, 35.2626);



    \path[draw=black,fill=black] (33.8631, 35.3526) -- (34.0222, 35.0143) -- (33.6871, 35.1794) -- (33.8631, 35.3526) -- (33.8631, 35.3526)-- cycle;;



    \node[anchor=south] (text6945) at (30.1597, 37.2597){c};



    \path[draw=black] (31.6346, 32.0675) ellipse (0.7299cm and 0.7299cm);



    \node[anchor=south] (text1331) at (31.6346, 31.9193){15};



    \path[draw=black] (29.3652, 30.638).. controls (29.6478, 30.776) and (29.9805, 30.9478) .. (30.2694, 31.1221).. controls (30.414, 31.2092) and (30.5633, 31.3073) .. (30.7075, 31.406);



    \path[draw=black,fill=black] (30.6264, 31.4999) -- (30.9859, 31.6032) -- (30.7693, 31.2984) -- (30.6264, 31.4999) -- (30.6264, 31.4999)-- cycle;;



    \node[anchor=south] (text6746) at (30.1597, 31.2702){c};



    \path[draw=black] (30.8945, 32.0929).. controls (28.9302, 32.1631) and (23.3811, 32.3624) .. (20.9564, 32.4492);



    \path[draw=black,fill=black] (20.9564, 32.3257) -- (20.6082, 32.4619) -- (20.9652, 32.5727) -- (20.9564, 32.3257) -- (20.9564, 32.3257)-- cycle;;



    \node[anchor=south] (text4751) at (25.7358, 32.4496){b};



    \path[draw=black] (32.348, 32.2538).. controls (32.6372, 32.3561) and (32.9653, 32.5046) .. (33.2186, 32.7096).. controls (33.5174, 32.9512) and (33.7739, 33.2839) .. (33.9743, 33.5968);



    \path[draw=black,fill=black] (33.8621, 33.6501) -- (34.1485, 33.8903) -- (34.0745, 33.5241) -- (33.8621, 33.6501) -- (33.8621, 33.6501)-- cycle;;



    \node[anchor=south] (text1525) at (33.1093, 32.8577){c};



    \path[draw=black] (19.0556, 36.0687).. controls (17.9828, 36.4296) and (15.8757, 37.139) .. (14.6004, 37.568);



    \path[draw=black,fill=black] (14.5634, 37.4502) -- (14.2684, 37.6798) -- (14.6424, 37.6844) -- (14.5634, 37.4502) -- (14.5634, 37.4502)-- cycle;;



    \node[anchor=south] (text5920) at (16.5996, 37.3055){a};



    \path[draw=black] (22.7591, 38.8056) ellipse (0.7299cm and 0.7299cm);



    \node[anchor=south] (text3664) at (22.7591, 38.6574){18};



    \path[draw=black] (20.4773, 38.1282).. controls (20.8315, 38.2351) and (21.2732, 38.3681) .. (21.6708, 38.4881);



    \path[draw=black,fill=black] (21.6246, 38.6031) -- (21.9978, 38.5868) -- (21.6958, 38.3667) -- (21.6246, 38.6031) -- (21.6246, 38.6031)-- cycle;;



    \node[anchor=south] (text6705) at (21.2845, 38.5392){c};



    \path[draw=black] (22.0356, 38.9791).. controls (21.8299, 39.0285) and (21.6034, 39.0804) .. (21.3939, 39.1231).. controls (20.1994, 39.3672) and (18.8256, 39.5958) .. (17.8622, 39.7478);



    \path[draw=black,fill=black] (17.8456, 39.6254) -- (17.5158, 39.8018) -- (17.8834, 39.8695) -- (17.8456, 39.6254) -- (17.8456, 39.6254)-- cycle;;



    \node[anchor=south] (text5843) at (19.7637, 39.708){c};



  \end{scope}

\end{tikzpicture}

    \end{adjustbox}
    \caption{Редуцированный автомат Глушкова}
    \label{fig:nfa}
\end{figure}

Таблица эквивалентности:

\begin{table}[h]
\begin{tabular}{|l|c|c|c|c|c|}
\hline
        & $\varepsilon$ & ba     & cb     & cc     & acb    \\ \hline
$\varepsilon$ & +          & -      & -      & -      & -      \\ \hline
a       & -          & +      & -      & -      & -      \\ \hline
aba     & +          & +      & +      & -      & -      \\ \hline
abab    & +          & -      & -      & +      & -      \\ \hline
ababab  & +          & -      & -      & +      & +      \\ \hline
\end{tabular}
\end{table}
\section{Переключающийся автомат}
Переключающийся автомат будет совпадать с НКА, так как нет нетривиальных инвариантов, которые можно будет проверять параллельно на всей строке. Проверка наличия подстрок aba или (b|ca) уже присутствует в НКА и является локально-последовательной.

\section{Расширенное регулярное выражение}
Исходное регулярное выражение:

$((aa|ab|cc)^*aba(aaa|bcc)^*)^*((abac|(cc)^*)(b|ca))^*$

Заметим, что подстроку $(aa|ab|cc)$ можно заменить на $(a[ab]|cc)$, а $[ab]$ можно поменять на дополнение $[\string^ c]$. Получим в итоге такое выражение:

$((a[\string^ c]|cc)^*aba(aaa|bcc)^*)^*((abac|(cc)^*)(b|ca))^*$

Далее, добавляем символы начала и конца выражения и получаем расширенное регулярное выражение:

$\string^ ((a[\string^ c]|cc)^*aba(aaa|bcc)^*)^*((abac|(cc)^*)(b|ca))^*$\$

\begin{itemize}
    \item{wildcard-операцию применить нигде не получится.}

    \item{Положительная итерация и опция не имеют смысла потому, что у нас нет ситуации $\tau \tau^*$}

    \item{Операции предпросмотра, ретроспективной проверки и их отрицательные версии тоже не имеют смысла, потому что ничего не упростят, а только усложнят регулярку.}
\end{itemize}

\end{document}
